\documentclass[11pt]{letter}
%\usepackage{german,a4}
\usepackage{german}

% Deutsche Umlaute als ASCII Zeichen eingeben:
% ASCII-Codetabellen werden dem Paket inputenc als Option �bergeben:
% Linux: latin1, Windows: ansinew, MS-DOS: cp850.
% Die Abh�ngigkeit vom Betriebssystem gilt aber nur f�r den Editor.
% Die LaTeX-Datei wird weiterhin auf jeder Plattform korrekt �bersetzt!
\usepackage[latin1]{inputenc}

%%\hyphenation{experience interacting acquired}
%%\parskip 0pt
%%\parindent 0pt
\setlength{\textwidth}{6in}
\setlength{\textheight}{10in}
\setlength{\hoffset}{0.25cm}
\setlength{\oddsidemargin}{0in}
\setlength{\topmargin}{0in}
\setlength{\voffset}{0in}
\setlength{\headheight}{0in}
\setlength{\headsep}{0in}
\pagestyle{empty}

\address{Dr.\ Wolfgang Tichy\\
Center for Gravitational Physics and Geometry\\
and Center for Gravitational Wave Physics\\
104 Davey Lab, Box 72 \\
The Pennsylvania State University \\
University Park, PA 16802}

\signature{Wolfgang Tichy}


\begin{document}
\begin{letter}{Universit�t Leipzig\\
Dekan der Fakult�t f�r Physik und Geowissenschaften\\
Linn\'estr. 5\\
04103 Leipzig}


\opening{Sehr geehrte Damen und Herren,}

hiermit m�chte ich mich f�r die von Ihnen ausgeschriebene 
C3-Professur im Bereich Theoretische Physik - Gravitationstheorie
bewerben.
Beigelegt finden Sie einen tabellarischen Lebenslauf und eine Beschreibung 
meiner Forschungsinteressen.
%In addition I have
%asked Bernd Br�gmann, Pablo Laguna, \'Eanna Flanagan and Abhay Ashtekar to
%send you letters of recommendation on my behalf.

Zur Zeit habe ich noch eine gemeinsame Postdoc-Stelle am Center for
Gravitational Physics and Geometry sowie am Center for Gravitational Wave
Physics an der Pennsylvania State University in den USA.
Ab August 2004 habe ich ein Stellenangebot als Assistant Professor of
Physics in der Spacetime Physics Group an der Florida Atlantic University
in den USA.
Mein Forschungsschwerpunkt ist im Gebiet der numerischen
Relativit�tstheorie, insbesondere arbeite ich an der numerischen Simulation 
von bin�ren Schwarzen L�chern.
Ich bin sowohl an der zeitlichen Entwicklung solcher Systeme als auch 
an der Konstruktion von realistischen Anfangsdaten interessiert.
Bis September 2002 war ich wissenschaftlicher Mitarbeiter
am Albert-Einstein-Institut in Golm,
wo ich haupts�chlich daran gearbeitet habe Post-Newtonsche
Anfangsdaten so zu erweitern, dass sie den Allgemein Relativistischen
Zwangsbedingungen gen�gen.
Vor meiner Zeit als Postdoc, war ich Doktorand an der Cornell University,
wo ich unter Professor \'Eanna Flanagan promoviert habe.
In Cornell habe ich mich mit
Gravitationswellen, Post-Newtonscher Theorie, Strahlungsr�ckwirkung 
und Semiklassischer Relativit�tstheorie besch�ftigt.

Ich bin stark daran interessiert eine wissenschaftliche Laufbahn einzuschlagen
und glaube, dass mir eine Stelle an der Universit�t Leipzig dazu
ausgezeichnete M�glichkeiten bieten w�rde. Unter anderem sehe ich gute
M�glichkeiten zur Zusammenarbeit mit der Fakult�t f�r Mathematik und
Informatik sowie mit dem MIS in Leipzig und dem AEI in Golm. Alle drei haben
Arbeitsgruppen im Gebiet Numerik, mit denen ich kooperieren k�nnte.

Sollten Sie bez�glich meiner Ausbildung oder meines wissenschaftlichen
Hintergrunds Fragen haben, k�nnen sie mich jederzeit in meinem B�ro
unter ++01 814 863 9595 erreichen. Ich w�rde mich freuen bald von Ihnen zu
h�ren.

\closing{Mit freundlichen Gr��en,}

\bigskip
%\bigskip 
%\bigskip

\encl{Lebenslauf mit Publikationsliste\\
Beschreibung der Forschungsinteressen}


\end{letter}
\end{document}



