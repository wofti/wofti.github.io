\documentclass[11pt]{article}

%Deutsche Umlaute als 8 Bit ISO-8859-1 Zeichen eingeben:
% ASCII-Codetabellen werden dem Paket inputenc als Option �bergeben:
% Linux: latin1, Windows: ansinew, MS-DOS: cp850. 
% Die Abh�ngigkeit vom Betriebssystem gilt aber nur f�r den Editor.
% Die LaTeX-Datei wird weiterhin auf jeder Plattform korrekt �bersetzt!
%Deutsche Umlaute als utf-8 Zeichen eingeben:
% \usepackage[utf8]{inputenc}  % <-- für Linux
% Die Abhängigkeit vom Betriebssystem gilt aber nur für den Editor.
% Die LaTeX-Datei wird weiterhin auf jeder Plattform korrekt übersetzt!
% z.B.: äöüß ÄÖÜ
%\usepackage[latin1]{inputenc}
\usepackage[utf8]{inputenc}

%\pagestyle{empty}

\topmargin      = 0in
\oddsidemargin  = 0in
\evensidemargin = 0in
\textwidth  = 6.5in
\textheight = 9in  % 8,8in
\headheight = 0in
\headsep    = 0in  % 0.2in
%\footskip   = 0in
\parindent=0pt
\parsep=0pt

\begin{document}

%%%%%%%%%%%%%%%%%%%%%%%%%%%%%%%%%%%%%%%%%%%%%%%%%%%
\begin{center}

\begin{tabular}{ll}
		&{\large \bf Wolfgang Tichy}\\\\
Geburtsdatum:	& 18. Juli 1969\\
Staatsangehörigkeit:   & deutsch\\\\
Adresse:	& Science \& Engineering Building, Room 432 \\
		& Physics Department\\
		& Florida Atlantic University\\
		& 777 Glades Road\\
		& Boca Raton, FL 33431\\\\

Telefon:	& (561) 297 3387\\
Fax:		& (561) 297 2662\\

E-Mail:		& wolf@fau.edu \\

Homepage: 	& http://www.physics.fau.edu/$\tilde{\ }$wolf \\


\end{tabular}

\end{center}


%%%%%%%%%%%%%%%%%%%%%%%%%%%%%%%%%%%%%%%%%%%%%%%%%%% 
\bigskip
{\bf Hochschulabschlüsse::}\\

\begin{tabular}{ll}
{2001}&  Ph.D. in Physics, Cornell University,
	  Ithaca, NY, USA \\
	& \\
%{\bf 1999}&  M.S. in Physics, Cornell University,
%	   Ithaca, NY, USA \\
%		& \\
{1996}&  Diplom in Physik, Universit\"at Karlsruhe,
	   Karlsruhe\\ 
	&\\
\end{tabular}

%%%%%%%%%%%%%%%%%%%%%%%%%%%%%%%%%%%%%%%%%%%%%%%%%%% 
\bigskip

{\bf Berufserfahrung:}\\

\begin{tabular}{ll}
{seit 2005}	& Assistant Professor of Physics\\
		& {Physics Department}\\
		& {Florida Atlantic University (FAU),}
		  {Boca Raton, FL 33431, USA}\\

{2002-2004}	& Wissenschaftlicher Mitarbeiter (Postdoc)\\
		& {Center for Gravitational Physics and Geometry} 
			\& {Center for Gravitational Wave Physics}\\
		& {Penn State University, }
		  {State College, PA, USA} \\

{2001-2002}	& Wissenschaftlicher Mitarbeiter (BAT IIa)\\
		& {Albert-Einstein-Institut,}
		 {Golm} \\

{1996-2001}	& Doktorand von Professor \'E. Flanagan \\
		& {Center for Radiophysics and Space Research,}
		  {Cornell University, Ithaca, NY, USA}\\

{1996-2001}	& Teaching Assistant\\
		& {Physics Department,}
		  {Cornell University, Ithaca, NY, USA}\\

{1995-1996}	& Diplomand von Professor Gerd Sch\"on,\\
       		& {Institut für Theoretische Festkörperphysik,} 
		  {Universit\"at Karlsruhe, Karlsruhe}\\

{1995-1996}	& Tutor in Physik \\
		& {Universit\"at Karlsruhe, Karlsruhe}\\
\end{tabular}

%%%%%%%%%%%%%%%%%%%%%%%%%%%%%%%%%%%%%%%%%%%%%%%%%%% 
%\vspace{0.3in}
%
%{\bf Course Work:}
%
%\vspace{0.25cm}
%In addition to standard graduate courses in Physics, I have also taken
%the following more advanced classes.
%
%\vspace{0.25cm}
%
%\begin{tabular}{lll}
%
%\underline{Physics} & \underline{Astronomy } & 
% \underline{Mathematics} \\
%& & \\
%Quantum Field Theory  & Cosmology         & Physics and Modern Geometry \\
%General Relativity    & Compact Objects   & Mathematical Methods in Physics\\
%Computational Physics & Galactic Structure& \\
%Gravitational Waves   & Stellar Structure & \\
%Statistical Physics   &                   & \\
%
%\end{tabular}

%%%%%%%%%%%%%%%%%%%%%%%%%%%%%%%%%%%%%%%%%%%%%%%%%%% 
\bigskip
%\bigskip
%\clearpage

{\bf Forschungsinteressen:}

\begin{itemize}
\item	Numerische Relativitätstheorie \& Gravitationswellen:
	\begin{itemize}
	\item	Numerische Simulation von Schwarzen Löchern und
		Neutronensternen
	\item	Binärsysteme als Quellen von Gravitationswellen
	\item	Konstruktion realistischer Anfangsdaten
                  für Schwarzen Löcher und Neutronensternen
	\item	Numerische Simulation unter Benutzung
                  pseudospektraler Methoden
	\item	Post-Newtonsche Anfangsdaten für Schwarzer Löcher
	\item	Teilchenbahnen um
                Schwarze Löcher unter Einschluss von
		Strahlungsrückwirkung
	\end{itemize}
\end{itemize}




%%%%%%%%%%%%%%%%%%%%%%%%%%%%%%%%%%%%%%%%%%%%%%%%%%%
\bigskip
\bigskip

{\bf Veröffentlichungen und Preprints:}\\


%%%%%%%%%%%%%%%%%%%%%%%%%%%%%%%%%%%%%%%%%%%%%%%%%%%%%%%%%%%%%%%%%
%
% Publication list, to be included with 
%%%%%%%%%%%%%%%%%%%%%%%%%%%%%%%%%%%%%%%%%%%%%%%%%%%%%%%%%%%%%%%%%
%
% Publication list, to be included with 
%%%%%%%%%%%%%%%%%%%%%%%%%%%%%%%%%%%%%%%%%%%%%%%%%%%%%%%%%%%%%%%%%
%
% Publication list, to be included with \input{publist.tex}
%
%%%%%%%%%%%%%%%%%%%%%%%%%%%%%%%%%%%%%%%%%%%%%%%%%%%%%%%%%%%%%%%%% 

%%  \underline{W. Tichy} and B. Br\"ugmann,
%%  ``Properties of a new strongly hyperbolic first order version of 
%%    the BSSN system'', 
%%  in preparation
%% \\
%
%%  \underline{W. Tichy} and \'E. \'E. Flanagan, 
%%  ``Computing the evolution of the Carter constant for scalar radiation from  
%%    a conserved quantity for scalar fields in Kerr spacetime'', 
%%  in preparation 
%% \\


L. Ji, A. Adhikari, \underline{W. Tichy},
``Toward Moving Puncture Simulations with the Generalized Harmonic System'',
in preparation
\\

G. Doulis, S. Bernuzzi, \underline{W. Tichy},
``Entropy based flux limiting scheme for conservation laws'',
accepted for publication in Phys. Rev. D,
arXiv:2401.04770 [gr-qc]
\\

R. Gamba, M. Breschi, S. Bernuzzi, A. Nagar, W. Cook, G. Doulis, F. Fabbri,
N. Ortiz, A. Poudel, A. Rashti, \underline{W. Tichy}, M. Ujevic,
``Analytically improved and numerical-relativity informed effective-one-body
model for coalescing binary neutron stars'',
submitted to Phys. Rev. D,
arXiv:2307.15125 [gr-qc]
\\

H. R. Rüter, V. Sagun, \underline{W. Tichy}, T. Dietrich,
``Quasi-equilibrium configurations of binary systems of dark matter
admixed neutron stars'',
Phys. Rev. {\bf D108}, 124080 (2023),
arXiv:2301.03568 [gr-qc]
\\

\underline{W. Tichy}, L. Ji, A. Adhikari, A. Rashti, M. Pirog,
``The new discontinuous Galerkin methods based numerical relativity
program Nmesh'',
Class. Quantum Grav. {\bf 40}, 025004 (2023),
arXiv:2212.06340 % [gr-qc]
\\

A. Gonzalez, F. Zappa, M. Breschi, S. Bernuzzi, D. Radice, A. Adhikari,
A. Camilletti, S. V. Chaurasia, G. Doulis, S. Padamata, A. Rashti,
M. Ujevic, B. Brügmann, W. Cook, T. Dietrich, A. Perego, A. Poudel,
\underline{W. Tichy},
``Second release of the CoRe database of binary neutron star merger
waveforms'',
Class. Quantum Grav. {\bf 40}, 085011 (2023),
arXiv:2210.16366 [gr-qc]
\\

T. Feghhi, \underline{W. Tichy}, and A. W. C. Lau,
``Pulling a harmonically bound particle subjected to Coulombic friction:
A nonequilibrium analysis'',
Phys. Rev. {\bf E106}, 024407 (2022)
\\

S. Mukherjee, N. K. Johnson-McDaniel, \underline{W. Tichy}, S. L. Liebling,
``Conformally curved initial data for charged, spinning black hole
binaries on arbitrary orbits'',
%under review??? I wonder if Soham has resubmitted it by now???
arXiv:2202.12133 [gr-qc]
\\

M. Ujevic, A. Rashti, H. Gieg, \underline{W. Tichy}, T. Dietrich,
``High-accuracy high-mass ratio simulations for binary neutron stars and
their comparison to existing waveform models'',
Phys. Rev. {\bf D106}, 023029 (2022),
arXiv:2202.09343 [gr-qc]
\\

A. Rashti, F. M. Fabbri, B. Brügmann, S. V. Chaurasia, T. Dietrich, M. Ujevic,
\underline{W. Tichy},
``New pseudospectral code for the construction of initial data'',
Phys. Rev. {\bf D105}, 104027 (2022),
arXiv:2109.14511 [gr-qc]
\\

R. Dudi, A. Adhikari, B. Brügmann, T. Dietrich, K. Hayashi, K. Kawaguchi,
K. Kiuchi, K. Kyutoku, M. Shibata, \underline{W. Tichy},
``Investigating GW190425 with numerical-relativity simulations'',
Phys. Rev. {\bf D106}, 084039 (2022),
arXiv:2109.04063  [gr-qc]
\\

R. Dudi, T. Dietrich, A. Rashti, B. Bruegmann, J. Steinhoff,
\underline{W. Tichy},
``High-accuracy simulations of highly spinning binary neutron star
systems'',
Phys. Rev. {\bf D105}, 064050 (2022),
arXiv:2108.10429  [gr-qc]
\\

A. Poudel, \underline{W. Tichy}, B. Brügmann, T. Dietrich,
``Increasing the accuracy of binary neutron star simulations with an
improved vacuum treatment'',
Phys. Rev. {\bf D102}, 104014 (2020),
arXiv:2009.06617 [gr-qc]
\\

S. V. Chaurasia, T. Dietrich, M. Ujevic, K. Hendriks, R. Dudi, F. M. Fabbri,
\underline{W. Tichy},B. Br\"ugmann,
``Gravitational waves and mass ejecta from binary neutron star mergers:
Effect of the spin orientation'',
Phys. Rev. {\bf D102}, 024087 (2020),
arXiv:2003.11901 [gr-qc]
\\

\underline{W. Tichy}, A. Rashti, T. Dietrich, R. Dudi, B. Br\"ugmann,
``Constructing Binary Neutron Star Initial Data with High Spins,
High Compactness, and High Mass-Ratios'',
Phys. Rev. {\bf D100}, 124046 (2019),
arXiv:1910.09690 [gr-qc]
\\

T. Dietrich, A. Samajdar, S. Khan, N.K. Johnson-McDaniel, R. Dudi, 
\underline{W. Tichy},
``Improving the NRTidal model for binary neutron star systems'',
Phys. Rev. {\bf D100}, 044003 (2019),
arXiv:1905.06011 [gr-qc]
\\

S.V. Chaurasia, T. Dietrich, N.K. Johnson-McDaniel, M. Ujevic,
\underline{W. Tichy}, B. Br\"ugmann,
``Gravitational waves and mass ejecta from binary neutron star mergers:
Effect of large eccentricities'',
Phys. Rev. {\bf D98}, 104005 (2018),
arXiv:1807.06857 [gr-qc]
\\

T. Dietrich, D. Radice, S. Bernuzzi, F. Zappa, A. Perego, B. Br\"ugmann,
S.V. Chaurasia, R. Dudi, \underline{W. Tichy}, M. Ujevic,
``CoRe database of binary neutron star merger waveforms'',
Class. Quantum Grav. {\bf 35} (2018) 24LT01,
arXiv:1806.01625 [gr-qc]
\\

T. Dietrich, S. Bernuzzi, B. Br\"ugmann, \underline{W. Tichy},
``High-resolution numerical relativity simulations of spinning binary
neutron star mergers'',
Proceeding for the 26th Euromicro International Conference on
Parallel, Distributed, and Network-Based Processing
in Cambridge 2018, PDP 2018 IEEE Catalog Number: CFP18169
arXiv:1803.07965 [gr-qc]
\\

T. Dietrich, S. Bernuzzi, B. Br\"ugmann, M. Ujevic, \underline{W. Tichy},
``Numerical Relativity Simulations of Precessing Binary Neutron Star Mergers'',
Phys. Rev. {\bf D97}, 064002, 2018,
arXiv:1712.02992 [gr-qc]
\\

T. Dietrich, S. Bernuzzi, \underline{W. Tichy},
``Closed-form tidal approximants for binary neutron star gravitational
waveforms constructed from high-resolution numerical relativity simulations'',
Phys. Rev. {\bf D96}, 121501, 2017,
arXiv:1706.02969 [gr-qc]
\\

T. Dietrich, S. Bernuzzi, M. Ujevic, \underline{W. Tichy},
``Gravitational waves and mass ejecta from binary neutron star mergers:
Effect of the stars' rotation'',
Phys. Rev. {\bf D95}, 044045, 2017,
arXiv:1611.07367 [gr-qc]
\\

\underline{W. Tichy},
``The initial value problem as it relates to numerical relativity'',
Rept. Prog. Phys. {\bf 80}, 026901, 2017,
arXiv:1610.03805 [gr-qc]
\\

T. Dietrich, M. Ujevic, \underline{W. Tichy}, S. Bernuzzi, B. Bruegmann,
``Gravitational waves and mass ejecta from binary neutron star mergers:
Effect of the mass-ratio'',
Phys. Rev. {\bf D95}, 024029, 2017,     
arXiv:1607.06636 [gr-qc]
\\

T. Dietrich, N. Moldenhauer, N. K. Johnson-McDaniel, S. Bernuzzi,
C. M. Markakis, B. Bruegmann, \underline{W. Tichy},
``Binary Neutron Stars with Generic Spin, Eccentricity, Mass ratio, and       
Compactness - Quasi-equilibrium Sequences and First Evolutions'',
Phys. Rev. {\bf D92}, 124007, 2015,
arXiv:1507.07100 [gr-qc]
\\

\underline{W. Tichy}, J. R. McDonald, W. A. Miller,
``New efficient algorithm for the isometric embedding of
2-surface metrics in 3 dimensional Euclidean space'',
Class. Quantum Grav. {\bf 32}, 015002 (2015),
arXiv:1411.0975 [gr-qc]
\\

N. Moldenhauer, C. M. Markakis, N. K. Johnson-McDaniel,
\underline{W. Tichy}, B. Bruegmann,
``Initial data for binary neutron stars with adjustable eccentricity'',
Phys. Rev. {\bf D90}, 084043, 2014,
arXiv:1408.4136 [gr-qc]
\\

J. Aasi et al., \underline{W. Tichy} [co-author],
``The NINJA-2 project: Detecting and characterizing gravitational waveforms
modelled using numerical binary black hole simulations'',
Class. Quantum Grav. {\bf 31}, 115004 (2014),
arXiv:1401.0939 [gr-qc]
\\

S. Bernuzzi, T. Dietrich, \underline{W. Tichy}, B. Bruegmann,
``Mergers of binary neutron stars with realistic spin'',
Phys. Rev. {\bf D89}, 104021, 2014,
arXiv:1311.4443 [gr-qc]
\\

I. Hinder, A. Buonanno, M. Boyle, Z. B. Etienne, J. Healy,
N. K. Johnson-McDaniel, A. Nagar, H. Nakano, Y. Pan, H. P. Pfeiffer,
M. Pürrer, C. Reisswig, M. A. Scheel, E. Schnetter, U. Sperhake,
B. Szilágyi, \underline{W. Tichy}, B. Wardell, A. Zenginoglu, D. Alic,
S. Bernuzzi, T. Bode, B. Brügmann, L. T. Buchman, M. Campanelli,
T. Chu, T. Damour, J. D. Grigsby, M. Hannam, R. Haas, D. A. Hemberger,
S. Husa, L. E. Kidder, P. Laguna, L. London, G. Lovelace, C. O. Lousto,
P. Marronetti, R. A. Matzner, P. Mösta, A. Mroué, D. Müller, B. C. Mundim,
A. Nerozzi, V. Paschalidis, D. Pollney, G. Reifenberger, L. Rezzolla,
S. L. Shapiro, D. Shoemaker, A. Taracchini, N. W. Taylor, S. A. Teukolsky,
M. Thierfelder, H. Witek, Y. Zlochower,
``Error-analysis and comparison to analytical models of numerical waveforms
produced by the NRAR Collaboration'',
Class. Quantum Grav. {\bf 31}, 025012 (2013), arXiv:1307.5307 [gr-qc]
\\

P Ajith, Michael Boyle, Duncan A Brown, Bernd Brügmann, Luisa T Buchman,
Laura Cadonati, Manuela Campanelli, Tony Chu, Zachariah B Etienne,
Stephen Fairhurst, Mark Hannam, James Healy, Ian Hinder, Sascha Husa,
Lawrence E Kidder, Badri Krishnan, Pablo Laguna, Yuk Tung Liu, Lionel London,
Carlos O Lousto, Geoffrey Lovelace, Ilana MacDonald, Pedro Marronetti,
Satya Mohapatra, Philipp Mösta, Doreen Müller, Bruno C Mundim,
Hiroyuki Nakano, Frank Ohme, Vasileios Paschalidis, Larne Pekowsky,
Denis Pollney, Harald P Pfeiffer, Marcelo Ponce, Michael Pürrer,
George Reifenberger, Christian Reisswig, Lucía Santamaría, Mark A Scheel,
Stuart L Shapiro, Deirdre Shoemaker, Carlos F Sopuerta, Ulrich Sperhake,
Béla Szilágyi, Nicholas W Taylor, Wolfgang Tichy,
Petr Tsatsin and Yosef Zlochower,
``Addendum to "The NINJA-2 catalog of hybrid
post-Newtonian/numerical-relativity waveforms for non-precessing black-hole
binaries"'',
Class. Quantum Grav. {\bf 30}, 199401 (2013)
\\

D. Hilditch, S. Bernuzzi, M. Thierfelder, Z. Cao,
\underline{W. Tichy}, B. Bruegmann,
``Compact binary evolutions with the Z4c formulation'',
Phys. Rev. {\bf D88},  084057, 2013,
arXiv:1212.2901 [gr-qc]
\\

\underline{W. Tichy},
``Constructing quasi-equilibrium initial data for binary neutron stars
with arbitrary spins'', Phys. Rev. {\bf D86}, 064024, 2012,
arXiv:1209.5336 [gr-qc]
\\

G. Reifenberger and \underline{W. Tichy},
``Alternatives to standard puncture initial data for binary black hole
evolution'', Phys. Rev. {\bf D86}, 064003, 2012,
arXiv:1205.5502 [gr-qc]
\\

%P. Ajith, Michael Boyle, Duncan A. Brown, Bernd Brügmann, Luisa T. Buchman,
%Laura Cadonati, Manuela Campanelli, Tony Chu, Zachariah B. Etienne, Stephen
%Fairhurst, Mark Hannam, James Healy, Ian Hinder, Sascha Husa, Lawrence E.
%Kidder, Badri Krishnan, Pablo Laguna, Yuk Tung Liu, Lionel London, Carlos O.
%Lousto, Geoffrey Lovelace, Ilana MacDonald, Pedro Marronetti, Satya
%Mohapatra, Philipp Mösta, Doreen Müller, Bruno C. Mundim, Hiroyuki Nakano,
%Frank Ohme, Vasileios Paschalidis, Larne Pekowsky, Denis Pollney, Harald P.
%Pfeiffer, Marcelo Ponce, Michael Pürrer, George Reifenberger, Christian
%Reisswig, Lucía Santamaría, Mark A. Scheel, Stuart L. Shapiro, Deirdre
%Shoemaker, Carlos F. Sopuerta, Ulrich Sperhake, Béla Szilágyi, Nicholas W.
%Taylor, Wolfgang Tichy,
%Petr Tsatsin, Yosef Zlochower
P. Ajith et al., \underline{W. Tichy} [co-author],
``The NINJA-2 catalog of hybrid post-Newtonian/numerical-relativity
waveforms for non-precessing black-hole binaries'',
Class. Quant. Grav. {\bf 29}, 124001 (2012), arXiv:1201.5319 [gr-qc]
\\

P. Marronetti and \underline{W. Tichy},
``Recent Advances in the Numerical Simulations of Binary Black Holes'',
Proceedings of the Department of Energy SciDAC Workshop (2011),
arXiv:1107.3703 [gr-qc]
\\

\underline{W. Tichy},
``Initial data for binary neutron stars with arbitrary spins '',
Phys. Rev. {\bf D84}, 024041, 2011,
arXiv:1107.1440 [gr-qc] 
\\

\underline{W. Tichy} and \'E. \'E. Flanagan,
``Covariant formulation of the post-1-Newtonian approximation to General
Relativity'', Phys. Rev. {\bf D 84}, 044038, 2011,
arXiv:1101.0588v1 [gr-qc]
\\

\underline{W. Tichy} and P. Marronetti,
``A Simple method to set up low eccentricity initial data for moving
puncture simulations'', Phys. Rev. {\bf D 83}, 024012, 2011,
arXiv:1010.2936v2 [gr-qc]
\\

B. J. Kelly, \underline{W. Tichy}, Y. Zlochower, M. Campanelli, B. F. Whiting, 
``Post-Newtonian Initial Data with Waves: Progress in Evolution'',
Class. Quant. Grav. {\bf 27}, 114005, 2010,  arXiv:0912.5311 [gr-qc]
\\

N. K. Johnson-McDaniel, N. Yunes, \underline{W. Tichy}, and B. J. Owen,
``Conformally curved binary black hole initial data including tidal
deformations and outgoing radiation'',
Phys. Rev. {\bf D 80}, 124039, 2009, arXiv:0907.0891 [gr-qc]
\\

\underline{W. Tichy},
``Long term black hole evolution with the BSSN system
by pseudo-spectral methods'',
Phys. Rev. {\bf D 80}, 104034, 2009, arXiv:0911.0973v2 [gr-qc] 
\\

I. Vega, S. Detweiler, P. Diener, and \underline{W. Tichy},
``Self-force with (3+1) codes: a primer for numerical relativists'',
Phys. Rev. {\bf D 80}, 084021, 2009, arXiv:0908.2138 [gr-qc]
\\

\underline{W. Tichy},
``A new numerical method to construct binary neutron star initial data'',
Class. Quant. Grav. {\bf 26}, 175018 (2009), arXiv:0908.0620 [gr-qc]
\\

% B. Aylott, J. G. Baker, W. D. Boggs, M. Boyle,
% P. R. Brady, D. A. Brown, B. Brügmann, L. T. Buchman,
% A. Buonanno, L. Cadonati, J. Camp, M. Campanelli, J. Centrella,
% S. Chatterji, N. Christensen, T. Chu, P. Diener, N. Dorband,
% Z. B. Etienne, J. Faber, S. Fairhurst, B. Farr,
% S. Fischetti, G. Guidi, L. M. Goggin, M. Hannam, 
% F. Herrmann, I. Hinder, S. Husa, V. Kalogera, D. Keppel,
% L. E. Kidder, B. J. Kelly, B. Krishnan, P. Laguna, C. O. Lousto,
% I. Mandel, P. Marronetti, R. Matzner, S. T. McWilliams,
% K. D. Matthews, R. Adam Mercer, S. R. P. Mohapatra, A. H. Mroué,
% H. Nakano, E. Ochsner, Y. Pan, L. Pekowsky,
% H. P. Pfeiffer, D. Pollney, F. Pretorius, V. Raymond,
% C. Reisswig, L. Rezzolla, O. Rinne, C. Robinson,
% C. Röver, L. Santamaría, B. Sathyaprakash, M. A. Scheel,
% E. Schnetter, J. Seiler, S. L. Shapiro, D. Shoemaker,
% U. Sperhake, A. Stroeer, R. Sturani, \underline{W. Tichy}, Y.
% T. Liu, M. van der Sluys, J. R. van Meter, R. Vaulin,
% A. Vecchio, J. Veitch, A. Viceré, J. T. Whelan, Y. Zlochower,
B. Aylott et al., \underline{W. Tichy} [co-author],
``Testing gravitational-wave searches with numerical relativity waveforms:
Results from the first Numerical INJection Analysis (NINJA) project'',
Class. Quant. Grav. {\bf 26}, 165008 (2009), arXiv:0901.4399 [gr-qc]
\\

L. Cadonati et al., \underline{W. Tichy} [co-author],
``Status of NINJA: the Numerical INJection Analysis project'',
Class. Quant. Grav. {\bf 26}, 114008 (2009), arXiv:0905.4227 [gr-qc]
\\

\underline{W. Tichy} and P. Marronetti,
``The final mass and spin of black hole mergers'',
Rapid Communication in Phys. Rev. {\bf D 78}, 081501 (2008),
arXiv:0807.2985 [gr-qc]
\\

P. Marronetti, \underline{W. Tichy}, B. Br\"ugmann, J. Gonzalez, 
and U. Sperhake, ``High-spin binary black hole mergers'',
Phys. Rev. {\bf D 77},064010 (2008), arXiv:0709.2160 [gr-qc]
\\

B. Br\"ugmann, J.A. Gonzalez, M. Hannam, S. Husa, U. Sperhake
and \underline{W. Tichy},
``Calibration of Moving Puncture Simulations'', 
Phys. Rev. {\bf D 77}, 024027 (2008), gr-qc/0610128
\\

B. J. Kelly, \underline{W. Tichy}, M. Campanelli and B. F. Whiting
``Black hole puncture initial data with realistic gravitational wave content'',
Phys. Rev. {\bf D 76}, 024008 (2007), arXiv:0704.0628 [gr-qc]
\\

\underline{W. Tichy} and P. Marronetti,
``Binary black hole mergers: large kicks for generic spin orientations'',
Rapid Communication in Phys. Rev. {\bf D 76}, 061502 (2007),
gr-qc/0703075
\\

P. Marronetti, \underline{W. Tichy}, B. Br\"ugmann, J. Gonzalez, M. Hannam, 
S. Husa, and U. Sperhake,
``Binary black holes on a budget: Simulations using workstations'',
Class. Quant. Grav. {\bf 24}, S43-S58 (2007), gr-qc/0701123
\\

% B. Br\"ugmann, J. Gonz{\'a}lez, M. Hannam, S. Husa, P. Marronetti, 
% U. Sperhake, and \underline{W. Tichy},
% {\em Gravitational Wave Signals from Simulations of Black Hole Dynamics},
% contribution to the 9th Results and Review Workshop of HLRS Computing Center, 
% Stuttgart, Germany,
% Oct. 13--14 2006, published in ``High Performance Computing in Science and 
% Engineering 2006'', Springer, 2006.
B. Br\"ugmann, J. Gonz{\'a}lez, M. Hannam, S. Husa, P. Marronetti, 
U. Sperhake, \underline{W. Tichy},
``Gravitational Wave Signals from Simulations of Black Hole Dynamics'',
contribution to the 9th Results and Review Workshop of HLRS Computing Center, 
Stuttgart, Germany,
Oct. 13--14 2006, published in ``High Performance Computing in Science and 
Engineering 2006'', Springer, 2006.
\\

N. Jansen, B. Br\"ugmann and \underline{W. Tichy},
``Numerical stability of the AA evolution system compared to the ADM and
BSSN systems'', Phys. Rev. {\bf D 74}, 084022 (2006)
\\

\underline{W. Tichy}, 
``Black hole evolution with the BSSN system by pseudo-spectral methods'',
Phys. Rev. {\bf D 74}, 084005 (2006), gr-qc/0609087
\\

N. Yunes, and \underline{W. Tichy}, 
``Improved initial data for black hole binaries by asymptotic matching
of post-Newtonian and perturbed black hole solutions'',
Phys. Rev. {\bf D 74}, 064013 (2006), gr-qc/0601046
\\

N. Yunes, \underline{W. Tichy}, B. J. Owen, and B. Br\"ugmann, 
``Binary black hole initial data from matched asymptotic expansions'',
Phys. Rev. {\bf D 74}, 104011 (2006), gr-qc/0503011.
\\

M. Ansorg, B. Br\"ugmann and \underline{W. Tichy},
``A single-domain spectral method for black hole puncture data'',
Phys. Rev. {\bf D 70}, 064011 (2004), gr-qc/0404056
\\

B. Br\"ugmann, \underline{W. Tichy} and N. Jansen,
``Numerical simulation of orbiting black holes'',
Phys. Rev. Lett. {\bf 92}, 211101 (2004), gr-qc/0312112
\\

\underline{W. Tichy} and B. Br\"ugmann,
``Quasi-equilibrium binary black hole sequences for 
puncture data derived from helical Killing vector conditions'', 
Phys. Rev. {\bf D 69}, 024006 (2004), gr-qc/0307027
\\

\underline{W. Tichy}, B. Br\"ugmann and P. Laguna, 
``Gauge conditions for binary black hole puncture data 
based on an approximate helical Killing vector'', 
Phys. Rev. {\bf D 68}, 064008 (2003), gr-qc/0306020
\\

\underline{W. Tichy}, B. Br\"ugmann, M. Campanelli and P. Diener, 
``Binary black hole initial data for numerical general
relativity based on post-Newtonian data'', 
Phys. Rev. {\bf D 67}, 064008 (2003), gr-qc/0207011
\\

\underline{W. Tichy} and \'E. \'E. Flanagan,
``Angular momentum ambiguities in asymptotically flat perturbed
stationary spacetimes'',
Proceedings of the Ninth Marcel Grossmann Meeting on
General Relativity, edited by V.G. Gurzadyan, R.T. Jantzen and R. Ruffini,
World Scientific, Singapore, p. 1622, (2002)
\\

\underline{W. Tichy} and \'E. \'E. Flanagan,
``Angular momentum ambiguities in asymptotically flat spacetimes which 
are perturbations of stationary spacetimes'', 
Class. Quant. Grav. {\bf 18}, 3995 (2001)
\\

\underline{W. Tichy}, \'E. \'E. Flanagan and E. Poisson,
``Can the post-Newtonian gravitational waveform of an inspiraling binary
be improved by solving the energy balance equation numerically?'',
Phys. Rev. D {\bf 61}, 104015 (2000), gr-qc/9912075 
\\

\underline{W. Tichy} and \'E. \'E. Flanagan, 
``How unique is the expected stress-energy tensor of a massive scalar
field?'',
Phys. Rev. D {\bf 58}, 124007 (1998), gr-qc/9807015 
\\

J. von Delft, D. S. Golubev, \underline{W. Tichy} and A. D. Zaikin,
``Parity-Effected Superconductivity in Ultrasmall Metallic Grains'',
Phys. Rev. Lett. {\bf 77}, 3189-3192 (1996), cond-mat/9604072 
\\

%
%%%%%%%%%%%%%%%%%%%%%%%%%%%%%%%%%%%%%%%%%%%%%%%%%%%%%%%%%%%%%%%%% 

%%  \underline{W. Tichy} and B. Br\"ugmann,
%%  ``Properties of a new strongly hyperbolic first order version of 
%%    the BSSN system'', 
%%  in preparation
%% \\
%
%%  \underline{W. Tichy} and \'E. \'E. Flanagan, 
%%  ``Computing the evolution of the Carter constant for scalar radiation from  
%%    a conserved quantity for scalar fields in Kerr spacetime'', 
%%  in preparation 
%% \\


L. Ji, A. Adhikari, \underline{W. Tichy},
``Toward Moving Puncture Simulations with the Generalized Harmonic System'',
in preparation
\\

G. Doulis, S. Bernuzzi, \underline{W. Tichy},
``Entropy based flux limiting scheme for conservation laws'',
accepted for publication in Phys. Rev. D,
arXiv:2401.04770 [gr-qc]
\\

R. Gamba, M. Breschi, S. Bernuzzi, A. Nagar, W. Cook, G. Doulis, F. Fabbri,
N. Ortiz, A. Poudel, A. Rashti, \underline{W. Tichy}, M. Ujevic,
``Analytically improved and numerical-relativity informed effective-one-body
model for coalescing binary neutron stars'',
submitted to Phys. Rev. D,
arXiv:2307.15125 [gr-qc]
\\

H. R. Rüter, V. Sagun, \underline{W. Tichy}, T. Dietrich,
``Quasi-equilibrium configurations of binary systems of dark matter
admixed neutron stars'',
Phys. Rev. {\bf D108}, 124080 (2023),
arXiv:2301.03568 [gr-qc]
\\

\underline{W. Tichy}, L. Ji, A. Adhikari, A. Rashti, M. Pirog,
``The new discontinuous Galerkin methods based numerical relativity
program Nmesh'',
Class. Quantum Grav. {\bf 40}, 025004 (2023),
arXiv:2212.06340 % [gr-qc]
\\

A. Gonzalez, F. Zappa, M. Breschi, S. Bernuzzi, D. Radice, A. Adhikari,
A. Camilletti, S. V. Chaurasia, G. Doulis, S. Padamata, A. Rashti,
M. Ujevic, B. Brügmann, W. Cook, T. Dietrich, A. Perego, A. Poudel,
\underline{W. Tichy},
``Second release of the CoRe database of binary neutron star merger
waveforms'',
Class. Quantum Grav. {\bf 40}, 085011 (2023),
arXiv:2210.16366 [gr-qc]
\\

T. Feghhi, \underline{W. Tichy}, and A. W. C. Lau,
``Pulling a harmonically bound particle subjected to Coulombic friction:
A nonequilibrium analysis'',
Phys. Rev. {\bf E106}, 024407 (2022)
\\

S. Mukherjee, N. K. Johnson-McDaniel, \underline{W. Tichy}, S. L. Liebling,
``Conformally curved initial data for charged, spinning black hole
binaries on arbitrary orbits'',
%under review??? I wonder if Soham has resubmitted it by now???
arXiv:2202.12133 [gr-qc]
\\

M. Ujevic, A. Rashti, H. Gieg, \underline{W. Tichy}, T. Dietrich,
``High-accuracy high-mass ratio simulations for binary neutron stars and
their comparison to existing waveform models'',
Phys. Rev. {\bf D106}, 023029 (2022),
arXiv:2202.09343 [gr-qc]
\\

A. Rashti, F. M. Fabbri, B. Brügmann, S. V. Chaurasia, T. Dietrich, M. Ujevic,
\underline{W. Tichy},
``New pseudospectral code for the construction of initial data'',
Phys. Rev. {\bf D105}, 104027 (2022),
arXiv:2109.14511 [gr-qc]
\\

R. Dudi, A. Adhikari, B. Brügmann, T. Dietrich, K. Hayashi, K. Kawaguchi,
K. Kiuchi, K. Kyutoku, M. Shibata, \underline{W. Tichy},
``Investigating GW190425 with numerical-relativity simulations'',
Phys. Rev. {\bf D106}, 084039 (2022),
arXiv:2109.04063  [gr-qc]
\\

R. Dudi, T. Dietrich, A. Rashti, B. Bruegmann, J. Steinhoff,
\underline{W. Tichy},
``High-accuracy simulations of highly spinning binary neutron star
systems'',
Phys. Rev. {\bf D105}, 064050 (2022),
arXiv:2108.10429  [gr-qc]
\\

A. Poudel, \underline{W. Tichy}, B. Brügmann, T. Dietrich,
``Increasing the accuracy of binary neutron star simulations with an
improved vacuum treatment'',
Phys. Rev. {\bf D102}, 104014 (2020),
arXiv:2009.06617 [gr-qc]
\\

S. V. Chaurasia, T. Dietrich, M. Ujevic, K. Hendriks, R. Dudi, F. M. Fabbri,
\underline{W. Tichy},B. Br\"ugmann,
``Gravitational waves and mass ejecta from binary neutron star mergers:
Effect of the spin orientation'',
Phys. Rev. {\bf D102}, 024087 (2020),
arXiv:2003.11901 [gr-qc]
\\

\underline{W. Tichy}, A. Rashti, T. Dietrich, R. Dudi, B. Br\"ugmann,
``Constructing Binary Neutron Star Initial Data with High Spins,
High Compactness, and High Mass-Ratios'',
Phys. Rev. {\bf D100}, 124046 (2019),
arXiv:1910.09690 [gr-qc]
\\

T. Dietrich, A. Samajdar, S. Khan, N.K. Johnson-McDaniel, R. Dudi, 
\underline{W. Tichy},
``Improving the NRTidal model for binary neutron star systems'',
Phys. Rev. {\bf D100}, 044003 (2019),
arXiv:1905.06011 [gr-qc]
\\

S.V. Chaurasia, T. Dietrich, N.K. Johnson-McDaniel, M. Ujevic,
\underline{W. Tichy}, B. Br\"ugmann,
``Gravitational waves and mass ejecta from binary neutron star mergers:
Effect of large eccentricities'',
Phys. Rev. {\bf D98}, 104005 (2018),
arXiv:1807.06857 [gr-qc]
\\

T. Dietrich, D. Radice, S. Bernuzzi, F. Zappa, A. Perego, B. Br\"ugmann,
S.V. Chaurasia, R. Dudi, \underline{W. Tichy}, M. Ujevic,
``CoRe database of binary neutron star merger waveforms'',
Class. Quantum Grav. {\bf 35} (2018) 24LT01,
arXiv:1806.01625 [gr-qc]
\\

T. Dietrich, S. Bernuzzi, B. Br\"ugmann, \underline{W. Tichy},
``High-resolution numerical relativity simulations of spinning binary
neutron star mergers'',
Proceeding for the 26th Euromicro International Conference on
Parallel, Distributed, and Network-Based Processing
in Cambridge 2018, PDP 2018 IEEE Catalog Number: CFP18169
arXiv:1803.07965 [gr-qc]
\\

T. Dietrich, S. Bernuzzi, B. Br\"ugmann, M. Ujevic, \underline{W. Tichy},
``Numerical Relativity Simulations of Precessing Binary Neutron Star Mergers'',
Phys. Rev. {\bf D97}, 064002, 2018,
arXiv:1712.02992 [gr-qc]
\\

T. Dietrich, S. Bernuzzi, \underline{W. Tichy},
``Closed-form tidal approximants for binary neutron star gravitational
waveforms constructed from high-resolution numerical relativity simulations'',
Phys. Rev. {\bf D96}, 121501, 2017,
arXiv:1706.02969 [gr-qc]
\\

T. Dietrich, S. Bernuzzi, M. Ujevic, \underline{W. Tichy},
``Gravitational waves and mass ejecta from binary neutron star mergers:
Effect of the stars' rotation'',
Phys. Rev. {\bf D95}, 044045, 2017,
arXiv:1611.07367 [gr-qc]
\\

\underline{W. Tichy},
``The initial value problem as it relates to numerical relativity'',
Rept. Prog. Phys. {\bf 80}, 026901, 2017,
arXiv:1610.03805 [gr-qc]
\\

T. Dietrich, M. Ujevic, \underline{W. Tichy}, S. Bernuzzi, B. Bruegmann,
``Gravitational waves and mass ejecta from binary neutron star mergers:
Effect of the mass-ratio'',
Phys. Rev. {\bf D95}, 024029, 2017,     
arXiv:1607.06636 [gr-qc]
\\

T. Dietrich, N. Moldenhauer, N. K. Johnson-McDaniel, S. Bernuzzi,
C. M. Markakis, B. Bruegmann, \underline{W. Tichy},
``Binary Neutron Stars with Generic Spin, Eccentricity, Mass ratio, and       
Compactness - Quasi-equilibrium Sequences and First Evolutions'',
Phys. Rev. {\bf D92}, 124007, 2015,
arXiv:1507.07100 [gr-qc]
\\

\underline{W. Tichy}, J. R. McDonald, W. A. Miller,
``New efficient algorithm for the isometric embedding of
2-surface metrics in 3 dimensional Euclidean space'',
Class. Quantum Grav. {\bf 32}, 015002 (2015),
arXiv:1411.0975 [gr-qc]
\\

N. Moldenhauer, C. M. Markakis, N. K. Johnson-McDaniel,
\underline{W. Tichy}, B. Bruegmann,
``Initial data for binary neutron stars with adjustable eccentricity'',
Phys. Rev. {\bf D90}, 084043, 2014,
arXiv:1408.4136 [gr-qc]
\\

J. Aasi et al., \underline{W. Tichy} [co-author],
``The NINJA-2 project: Detecting and characterizing gravitational waveforms
modelled using numerical binary black hole simulations'',
Class. Quantum Grav. {\bf 31}, 115004 (2014),
arXiv:1401.0939 [gr-qc]
\\

S. Bernuzzi, T. Dietrich, \underline{W. Tichy}, B. Bruegmann,
``Mergers of binary neutron stars with realistic spin'',
Phys. Rev. {\bf D89}, 104021, 2014,
arXiv:1311.4443 [gr-qc]
\\

I. Hinder, A. Buonanno, M. Boyle, Z. B. Etienne, J. Healy,
N. K. Johnson-McDaniel, A. Nagar, H. Nakano, Y. Pan, H. P. Pfeiffer,
M. Pürrer, C. Reisswig, M. A. Scheel, E. Schnetter, U. Sperhake,
B. Szilágyi, \underline{W. Tichy}, B. Wardell, A. Zenginoglu, D. Alic,
S. Bernuzzi, T. Bode, B. Brügmann, L. T. Buchman, M. Campanelli,
T. Chu, T. Damour, J. D. Grigsby, M. Hannam, R. Haas, D. A. Hemberger,
S. Husa, L. E. Kidder, P. Laguna, L. London, G. Lovelace, C. O. Lousto,
P. Marronetti, R. A. Matzner, P. Mösta, A. Mroué, D. Müller, B. C. Mundim,
A. Nerozzi, V. Paschalidis, D. Pollney, G. Reifenberger, L. Rezzolla,
S. L. Shapiro, D. Shoemaker, A. Taracchini, N. W. Taylor, S. A. Teukolsky,
M. Thierfelder, H. Witek, Y. Zlochower,
``Error-analysis and comparison to analytical models of numerical waveforms
produced by the NRAR Collaboration'',
Class. Quantum Grav. {\bf 31}, 025012 (2013), arXiv:1307.5307 [gr-qc]
\\

P Ajith, Michael Boyle, Duncan A Brown, Bernd Brügmann, Luisa T Buchman,
Laura Cadonati, Manuela Campanelli, Tony Chu, Zachariah B Etienne,
Stephen Fairhurst, Mark Hannam, James Healy, Ian Hinder, Sascha Husa,
Lawrence E Kidder, Badri Krishnan, Pablo Laguna, Yuk Tung Liu, Lionel London,
Carlos O Lousto, Geoffrey Lovelace, Ilana MacDonald, Pedro Marronetti,
Satya Mohapatra, Philipp Mösta, Doreen Müller, Bruno C Mundim,
Hiroyuki Nakano, Frank Ohme, Vasileios Paschalidis, Larne Pekowsky,
Denis Pollney, Harald P Pfeiffer, Marcelo Ponce, Michael Pürrer,
George Reifenberger, Christian Reisswig, Lucía Santamaría, Mark A Scheel,
Stuart L Shapiro, Deirdre Shoemaker, Carlos F Sopuerta, Ulrich Sperhake,
Béla Szilágyi, Nicholas W Taylor, Wolfgang Tichy,
Petr Tsatsin and Yosef Zlochower,
``Addendum to "The NINJA-2 catalog of hybrid
post-Newtonian/numerical-relativity waveforms for non-precessing black-hole
binaries"'',
Class. Quantum Grav. {\bf 30}, 199401 (2013)
\\

D. Hilditch, S. Bernuzzi, M. Thierfelder, Z. Cao,
\underline{W. Tichy}, B. Bruegmann,
``Compact binary evolutions with the Z4c formulation'',
Phys. Rev. {\bf D88},  084057, 2013,
arXiv:1212.2901 [gr-qc]
\\

\underline{W. Tichy},
``Constructing quasi-equilibrium initial data for binary neutron stars
with arbitrary spins'', Phys. Rev. {\bf D86}, 064024, 2012,
arXiv:1209.5336 [gr-qc]
\\

G. Reifenberger and \underline{W. Tichy},
``Alternatives to standard puncture initial data for binary black hole
evolution'', Phys. Rev. {\bf D86}, 064003, 2012,
arXiv:1205.5502 [gr-qc]
\\

%P. Ajith, Michael Boyle, Duncan A. Brown, Bernd Brügmann, Luisa T. Buchman,
%Laura Cadonati, Manuela Campanelli, Tony Chu, Zachariah B. Etienne, Stephen
%Fairhurst, Mark Hannam, James Healy, Ian Hinder, Sascha Husa, Lawrence E.
%Kidder, Badri Krishnan, Pablo Laguna, Yuk Tung Liu, Lionel London, Carlos O.
%Lousto, Geoffrey Lovelace, Ilana MacDonald, Pedro Marronetti, Satya
%Mohapatra, Philipp Mösta, Doreen Müller, Bruno C. Mundim, Hiroyuki Nakano,
%Frank Ohme, Vasileios Paschalidis, Larne Pekowsky, Denis Pollney, Harald P.
%Pfeiffer, Marcelo Ponce, Michael Pürrer, George Reifenberger, Christian
%Reisswig, Lucía Santamaría, Mark A. Scheel, Stuart L. Shapiro, Deirdre
%Shoemaker, Carlos F. Sopuerta, Ulrich Sperhake, Béla Szilágyi, Nicholas W.
%Taylor, Wolfgang Tichy,
%Petr Tsatsin, Yosef Zlochower
P. Ajith et al., \underline{W. Tichy} [co-author],
``The NINJA-2 catalog of hybrid post-Newtonian/numerical-relativity
waveforms for non-precessing black-hole binaries'',
Class. Quant. Grav. {\bf 29}, 124001 (2012), arXiv:1201.5319 [gr-qc]
\\

P. Marronetti and \underline{W. Tichy},
``Recent Advances in the Numerical Simulations of Binary Black Holes'',
Proceedings of the Department of Energy SciDAC Workshop (2011),
arXiv:1107.3703 [gr-qc]
\\

\underline{W. Tichy},
``Initial data for binary neutron stars with arbitrary spins '',
Phys. Rev. {\bf D84}, 024041, 2011,
arXiv:1107.1440 [gr-qc] 
\\

\underline{W. Tichy} and \'E. \'E. Flanagan,
``Covariant formulation of the post-1-Newtonian approximation to General
Relativity'', Phys. Rev. {\bf D 84}, 044038, 2011,
arXiv:1101.0588v1 [gr-qc]
\\

\underline{W. Tichy} and P. Marronetti,
``A Simple method to set up low eccentricity initial data for moving
puncture simulations'', Phys. Rev. {\bf D 83}, 024012, 2011,
arXiv:1010.2936v2 [gr-qc]
\\

B. J. Kelly, \underline{W. Tichy}, Y. Zlochower, M. Campanelli, B. F. Whiting, 
``Post-Newtonian Initial Data with Waves: Progress in Evolution'',
Class. Quant. Grav. {\bf 27}, 114005, 2010,  arXiv:0912.5311 [gr-qc]
\\

N. K. Johnson-McDaniel, N. Yunes, \underline{W. Tichy}, and B. J. Owen,
``Conformally curved binary black hole initial data including tidal
deformations and outgoing radiation'',
Phys. Rev. {\bf D 80}, 124039, 2009, arXiv:0907.0891 [gr-qc]
\\

\underline{W. Tichy},
``Long term black hole evolution with the BSSN system
by pseudo-spectral methods'',
Phys. Rev. {\bf D 80}, 104034, 2009, arXiv:0911.0973v2 [gr-qc] 
\\

I. Vega, S. Detweiler, P. Diener, and \underline{W. Tichy},
``Self-force with (3+1) codes: a primer for numerical relativists'',
Phys. Rev. {\bf D 80}, 084021, 2009, arXiv:0908.2138 [gr-qc]
\\

\underline{W. Tichy},
``A new numerical method to construct binary neutron star initial data'',
Class. Quant. Grav. {\bf 26}, 175018 (2009), arXiv:0908.0620 [gr-qc]
\\

% B. Aylott, J. G. Baker, W. D. Boggs, M. Boyle,
% P. R. Brady, D. A. Brown, B. Brügmann, L. T. Buchman,
% A. Buonanno, L. Cadonati, J. Camp, M. Campanelli, J. Centrella,
% S. Chatterji, N. Christensen, T. Chu, P. Diener, N. Dorband,
% Z. B. Etienne, J. Faber, S. Fairhurst, B. Farr,
% S. Fischetti, G. Guidi, L. M. Goggin, M. Hannam, 
% F. Herrmann, I. Hinder, S. Husa, V. Kalogera, D. Keppel,
% L. E. Kidder, B. J. Kelly, B. Krishnan, P. Laguna, C. O. Lousto,
% I. Mandel, P. Marronetti, R. Matzner, S. T. McWilliams,
% K. D. Matthews, R. Adam Mercer, S. R. P. Mohapatra, A. H. Mroué,
% H. Nakano, E. Ochsner, Y. Pan, L. Pekowsky,
% H. P. Pfeiffer, D. Pollney, F. Pretorius, V. Raymond,
% C. Reisswig, L. Rezzolla, O. Rinne, C. Robinson,
% C. Röver, L. Santamaría, B. Sathyaprakash, M. A. Scheel,
% E. Schnetter, J. Seiler, S. L. Shapiro, D. Shoemaker,
% U. Sperhake, A. Stroeer, R. Sturani, \underline{W. Tichy}, Y.
% T. Liu, M. van der Sluys, J. R. van Meter, R. Vaulin,
% A. Vecchio, J. Veitch, A. Viceré, J. T. Whelan, Y. Zlochower,
B. Aylott et al., \underline{W. Tichy} [co-author],
``Testing gravitational-wave searches with numerical relativity waveforms:
Results from the first Numerical INJection Analysis (NINJA) project'',
Class. Quant. Grav. {\bf 26}, 165008 (2009), arXiv:0901.4399 [gr-qc]
\\

L. Cadonati et al., \underline{W. Tichy} [co-author],
``Status of NINJA: the Numerical INJection Analysis project'',
Class. Quant. Grav. {\bf 26}, 114008 (2009), arXiv:0905.4227 [gr-qc]
\\

\underline{W. Tichy} and P. Marronetti,
``The final mass and spin of black hole mergers'',
Rapid Communication in Phys. Rev. {\bf D 78}, 081501 (2008),
arXiv:0807.2985 [gr-qc]
\\

P. Marronetti, \underline{W. Tichy}, B. Br\"ugmann, J. Gonzalez, 
and U. Sperhake, ``High-spin binary black hole mergers'',
Phys. Rev. {\bf D 77},064010 (2008), arXiv:0709.2160 [gr-qc]
\\

B. Br\"ugmann, J.A. Gonzalez, M. Hannam, S. Husa, U. Sperhake
and \underline{W. Tichy},
``Calibration of Moving Puncture Simulations'', 
Phys. Rev. {\bf D 77}, 024027 (2008), gr-qc/0610128
\\

B. J. Kelly, \underline{W. Tichy}, M. Campanelli and B. F. Whiting
``Black hole puncture initial data with realistic gravitational wave content'',
Phys. Rev. {\bf D 76}, 024008 (2007), arXiv:0704.0628 [gr-qc]
\\

\underline{W. Tichy} and P. Marronetti,
``Binary black hole mergers: large kicks for generic spin orientations'',
Rapid Communication in Phys. Rev. {\bf D 76}, 061502 (2007),
gr-qc/0703075
\\

P. Marronetti, \underline{W. Tichy}, B. Br\"ugmann, J. Gonzalez, M. Hannam, 
S. Husa, and U. Sperhake,
``Binary black holes on a budget: Simulations using workstations'',
Class. Quant. Grav. {\bf 24}, S43-S58 (2007), gr-qc/0701123
\\

% B. Br\"ugmann, J. Gonz{\'a}lez, M. Hannam, S. Husa, P. Marronetti, 
% U. Sperhake, and \underline{W. Tichy},
% {\em Gravitational Wave Signals from Simulations of Black Hole Dynamics},
% contribution to the 9th Results and Review Workshop of HLRS Computing Center, 
% Stuttgart, Germany,
% Oct. 13--14 2006, published in ``High Performance Computing in Science and 
% Engineering 2006'', Springer, 2006.
B. Br\"ugmann, J. Gonz{\'a}lez, M. Hannam, S. Husa, P. Marronetti, 
U. Sperhake, \underline{W. Tichy},
``Gravitational Wave Signals from Simulations of Black Hole Dynamics'',
contribution to the 9th Results and Review Workshop of HLRS Computing Center, 
Stuttgart, Germany,
Oct. 13--14 2006, published in ``High Performance Computing in Science and 
Engineering 2006'', Springer, 2006.
\\

N. Jansen, B. Br\"ugmann and \underline{W. Tichy},
``Numerical stability of the AA evolution system compared to the ADM and
BSSN systems'', Phys. Rev. {\bf D 74}, 084022 (2006)
\\

\underline{W. Tichy}, 
``Black hole evolution with the BSSN system by pseudo-spectral methods'',
Phys. Rev. {\bf D 74}, 084005 (2006), gr-qc/0609087
\\

N. Yunes, and \underline{W. Tichy}, 
``Improved initial data for black hole binaries by asymptotic matching
of post-Newtonian and perturbed black hole solutions'',
Phys. Rev. {\bf D 74}, 064013 (2006), gr-qc/0601046
\\

N. Yunes, \underline{W. Tichy}, B. J. Owen, and B. Br\"ugmann, 
``Binary black hole initial data from matched asymptotic expansions'',
Phys. Rev. {\bf D 74}, 104011 (2006), gr-qc/0503011.
\\

M. Ansorg, B. Br\"ugmann and \underline{W. Tichy},
``A single-domain spectral method for black hole puncture data'',
Phys. Rev. {\bf D 70}, 064011 (2004), gr-qc/0404056
\\

B. Br\"ugmann, \underline{W. Tichy} and N. Jansen,
``Numerical simulation of orbiting black holes'',
Phys. Rev. Lett. {\bf 92}, 211101 (2004), gr-qc/0312112
\\

\underline{W. Tichy} and B. Br\"ugmann,
``Quasi-equilibrium binary black hole sequences for 
puncture data derived from helical Killing vector conditions'', 
Phys. Rev. {\bf D 69}, 024006 (2004), gr-qc/0307027
\\

\underline{W. Tichy}, B. Br\"ugmann and P. Laguna, 
``Gauge conditions for binary black hole puncture data 
based on an approximate helical Killing vector'', 
Phys. Rev. {\bf D 68}, 064008 (2003), gr-qc/0306020
\\

\underline{W. Tichy}, B. Br\"ugmann, M. Campanelli and P. Diener, 
``Binary black hole initial data for numerical general
relativity based on post-Newtonian data'', 
Phys. Rev. {\bf D 67}, 064008 (2003), gr-qc/0207011
\\

\underline{W. Tichy} and \'E. \'E. Flanagan,
``Angular momentum ambiguities in asymptotically flat perturbed
stationary spacetimes'',
Proceedings of the Ninth Marcel Grossmann Meeting on
General Relativity, edited by V.G. Gurzadyan, R.T. Jantzen and R. Ruffini,
World Scientific, Singapore, p. 1622, (2002)
\\

\underline{W. Tichy} and \'E. \'E. Flanagan,
``Angular momentum ambiguities in asymptotically flat spacetimes which 
are perturbations of stationary spacetimes'', 
Class. Quant. Grav. {\bf 18}, 3995 (2001)
\\

\underline{W. Tichy}, \'E. \'E. Flanagan and E. Poisson,
``Can the post-Newtonian gravitational waveform of an inspiraling binary
be improved by solving the energy balance equation numerically?'',
Phys. Rev. D {\bf 61}, 104015 (2000), gr-qc/9912075 
\\

\underline{W. Tichy} and \'E. \'E. Flanagan, 
``How unique is the expected stress-energy tensor of a massive scalar
field?'',
Phys. Rev. D {\bf 58}, 124007 (1998), gr-qc/9807015 
\\

J. von Delft, D. S. Golubev, \underline{W. Tichy} and A. D. Zaikin,
``Parity-Effected Superconductivity in Ultrasmall Metallic Grains'',
Phys. Rev. Lett. {\bf 77}, 3189-3192 (1996), cond-mat/9604072 
\\

%
%%%%%%%%%%%%%%%%%%%%%%%%%%%%%%%%%%%%%%%%%%%%%%%%%%%%%%%%%%%%%%%%% 

%%  \underline{W. Tichy} and B. Br\"ugmann,
%%  ``Properties of a new strongly hyperbolic first order version of 
%%    the BSSN system'', 
%%  in preparation
%% \\
%
%%  \underline{W. Tichy} and \'E. \'E. Flanagan, 
%%  ``Computing the evolution of the Carter constant for scalar radiation from  
%%    a conserved quantity for scalar fields in Kerr spacetime'', 
%%  in preparation 
%% \\


L. Ji, A. Adhikari, \underline{W. Tichy},
``Toward Moving Puncture Simulations with the Generalized Harmonic System'',
in preparation
\\

G. Doulis, S. Bernuzzi, \underline{W. Tichy},
``Entropy based flux limiting scheme for conservation laws'',
accepted for publication in Phys. Rev. D,
arXiv:2401.04770 [gr-qc]
\\

R. Gamba, M. Breschi, S. Bernuzzi, A. Nagar, W. Cook, G. Doulis, F. Fabbri,
N. Ortiz, A. Poudel, A. Rashti, \underline{W. Tichy}, M. Ujevic,
``Analytically improved and numerical-relativity informed effective-one-body
model for coalescing binary neutron stars'',
submitted to Phys. Rev. D,
arXiv:2307.15125 [gr-qc]
\\

H. R. Rüter, V. Sagun, \underline{W. Tichy}, T. Dietrich,
``Quasi-equilibrium configurations of binary systems of dark matter
admixed neutron stars'',
Phys. Rev. {\bf D108}, 124080 (2023),
arXiv:2301.03568 [gr-qc]
\\

\underline{W. Tichy}, L. Ji, A. Adhikari, A. Rashti, M. Pirog,
``The new discontinuous Galerkin methods based numerical relativity
program Nmesh'',
Class. Quantum Grav. {\bf 40}, 025004 (2023),
arXiv:2212.06340 % [gr-qc]
\\

A. Gonzalez, F. Zappa, M. Breschi, S. Bernuzzi, D. Radice, A. Adhikari,
A. Camilletti, S. V. Chaurasia, G. Doulis, S. Padamata, A. Rashti,
M. Ujevic, B. Brügmann, W. Cook, T. Dietrich, A. Perego, A. Poudel,
\underline{W. Tichy},
``Second release of the CoRe database of binary neutron star merger
waveforms'',
Class. Quantum Grav. {\bf 40}, 085011 (2023),
arXiv:2210.16366 [gr-qc]
\\

T. Feghhi, \underline{W. Tichy}, and A. W. C. Lau,
``Pulling a harmonically bound particle subjected to Coulombic friction:
A nonequilibrium analysis'',
Phys. Rev. {\bf E106}, 024407 (2022)
\\

S. Mukherjee, N. K. Johnson-McDaniel, \underline{W. Tichy}, S. L. Liebling,
``Conformally curved initial data for charged, spinning black hole
binaries on arbitrary orbits'',
%under review??? I wonder if Soham has resubmitted it by now???
arXiv:2202.12133 [gr-qc]
\\

M. Ujevic, A. Rashti, H. Gieg, \underline{W. Tichy}, T. Dietrich,
``High-accuracy high-mass ratio simulations for binary neutron stars and
their comparison to existing waveform models'',
Phys. Rev. {\bf D106}, 023029 (2022),
arXiv:2202.09343 [gr-qc]
\\

A. Rashti, F. M. Fabbri, B. Brügmann, S. V. Chaurasia, T. Dietrich, M. Ujevic,
\underline{W. Tichy},
``New pseudospectral code for the construction of initial data'',
Phys. Rev. {\bf D105}, 104027 (2022),
arXiv:2109.14511 [gr-qc]
\\

R. Dudi, A. Adhikari, B. Brügmann, T. Dietrich, K. Hayashi, K. Kawaguchi,
K. Kiuchi, K. Kyutoku, M. Shibata, \underline{W. Tichy},
``Investigating GW190425 with numerical-relativity simulations'',
Phys. Rev. {\bf D106}, 084039 (2022),
arXiv:2109.04063  [gr-qc]
\\

R. Dudi, T. Dietrich, A. Rashti, B. Bruegmann, J. Steinhoff,
\underline{W. Tichy},
``High-accuracy simulations of highly spinning binary neutron star
systems'',
Phys. Rev. {\bf D105}, 064050 (2022),
arXiv:2108.10429  [gr-qc]
\\

A. Poudel, \underline{W. Tichy}, B. Brügmann, T. Dietrich,
``Increasing the accuracy of binary neutron star simulations with an
improved vacuum treatment'',
Phys. Rev. {\bf D102}, 104014 (2020),
arXiv:2009.06617 [gr-qc]
\\

S. V. Chaurasia, T. Dietrich, M. Ujevic, K. Hendriks, R. Dudi, F. M. Fabbri,
\underline{W. Tichy},B. Br\"ugmann,
``Gravitational waves and mass ejecta from binary neutron star mergers:
Effect of the spin orientation'',
Phys. Rev. {\bf D102}, 024087 (2020),
arXiv:2003.11901 [gr-qc]
\\

\underline{W. Tichy}, A. Rashti, T. Dietrich, R. Dudi, B. Br\"ugmann,
``Constructing Binary Neutron Star Initial Data with High Spins,
High Compactness, and High Mass-Ratios'',
Phys. Rev. {\bf D100}, 124046 (2019),
arXiv:1910.09690 [gr-qc]
\\

T. Dietrich, A. Samajdar, S. Khan, N.K. Johnson-McDaniel, R. Dudi, 
\underline{W. Tichy},
``Improving the NRTidal model for binary neutron star systems'',
Phys. Rev. {\bf D100}, 044003 (2019),
arXiv:1905.06011 [gr-qc]
\\

S.V. Chaurasia, T. Dietrich, N.K. Johnson-McDaniel, M. Ujevic,
\underline{W. Tichy}, B. Br\"ugmann,
``Gravitational waves and mass ejecta from binary neutron star mergers:
Effect of large eccentricities'',
Phys. Rev. {\bf D98}, 104005 (2018),
arXiv:1807.06857 [gr-qc]
\\

T. Dietrich, D. Radice, S. Bernuzzi, F. Zappa, A. Perego, B. Br\"ugmann,
S.V. Chaurasia, R. Dudi, \underline{W. Tichy}, M. Ujevic,
``CoRe database of binary neutron star merger waveforms'',
Class. Quantum Grav. {\bf 35} (2018) 24LT01,
arXiv:1806.01625 [gr-qc]
\\

T. Dietrich, S. Bernuzzi, B. Br\"ugmann, \underline{W. Tichy},
``High-resolution numerical relativity simulations of spinning binary
neutron star mergers'',
Proceeding for the 26th Euromicro International Conference on
Parallel, Distributed, and Network-Based Processing
in Cambridge 2018, PDP 2018 IEEE Catalog Number: CFP18169
arXiv:1803.07965 [gr-qc]
\\

T. Dietrich, S. Bernuzzi, B. Br\"ugmann, M. Ujevic, \underline{W. Tichy},
``Numerical Relativity Simulations of Precessing Binary Neutron Star Mergers'',
Phys. Rev. {\bf D97}, 064002, 2018,
arXiv:1712.02992 [gr-qc]
\\

T. Dietrich, S. Bernuzzi, \underline{W. Tichy},
``Closed-form tidal approximants for binary neutron star gravitational
waveforms constructed from high-resolution numerical relativity simulations'',
Phys. Rev. {\bf D96}, 121501, 2017,
arXiv:1706.02969 [gr-qc]
\\

T. Dietrich, S. Bernuzzi, M. Ujevic, \underline{W. Tichy},
``Gravitational waves and mass ejecta from binary neutron star mergers:
Effect of the stars' rotation'',
Phys. Rev. {\bf D95}, 044045, 2017,
arXiv:1611.07367 [gr-qc]
\\

\underline{W. Tichy},
``The initial value problem as it relates to numerical relativity'',
Rept. Prog. Phys. {\bf 80}, 026901, 2017,
arXiv:1610.03805 [gr-qc]
\\

T. Dietrich, M. Ujevic, \underline{W. Tichy}, S. Bernuzzi, B. Bruegmann,
``Gravitational waves and mass ejecta from binary neutron star mergers:
Effect of the mass-ratio'',
Phys. Rev. {\bf D95}, 024029, 2017,     
arXiv:1607.06636 [gr-qc]
\\

T. Dietrich, N. Moldenhauer, N. K. Johnson-McDaniel, S. Bernuzzi,
C. M. Markakis, B. Bruegmann, \underline{W. Tichy},
``Binary Neutron Stars with Generic Spin, Eccentricity, Mass ratio, and       
Compactness - Quasi-equilibrium Sequences and First Evolutions'',
Phys. Rev. {\bf D92}, 124007, 2015,
arXiv:1507.07100 [gr-qc]
\\

\underline{W. Tichy}, J. R. McDonald, W. A. Miller,
``New efficient algorithm for the isometric embedding of
2-surface metrics in 3 dimensional Euclidean space'',
Class. Quantum Grav. {\bf 32}, 015002 (2015),
arXiv:1411.0975 [gr-qc]
\\

N. Moldenhauer, C. M. Markakis, N. K. Johnson-McDaniel,
\underline{W. Tichy}, B. Bruegmann,
``Initial data for binary neutron stars with adjustable eccentricity'',
Phys. Rev. {\bf D90}, 084043, 2014,
arXiv:1408.4136 [gr-qc]
\\

J. Aasi et al., \underline{W. Tichy} [co-author],
``The NINJA-2 project: Detecting and characterizing gravitational waveforms
modelled using numerical binary black hole simulations'',
Class. Quantum Grav. {\bf 31}, 115004 (2014),
arXiv:1401.0939 [gr-qc]
\\

S. Bernuzzi, T. Dietrich, \underline{W. Tichy}, B. Bruegmann,
``Mergers of binary neutron stars with realistic spin'',
Phys. Rev. {\bf D89}, 104021, 2014,
arXiv:1311.4443 [gr-qc]
\\

I. Hinder, A. Buonanno, M. Boyle, Z. B. Etienne, J. Healy,
N. K. Johnson-McDaniel, A. Nagar, H. Nakano, Y. Pan, H. P. Pfeiffer,
M. Pürrer, C. Reisswig, M. A. Scheel, E. Schnetter, U. Sperhake,
B. Szilágyi, \underline{W. Tichy}, B. Wardell, A. Zenginoglu, D. Alic,
S. Bernuzzi, T. Bode, B. Brügmann, L. T. Buchman, M. Campanelli,
T. Chu, T. Damour, J. D. Grigsby, M. Hannam, R. Haas, D. A. Hemberger,
S. Husa, L. E. Kidder, P. Laguna, L. London, G. Lovelace, C. O. Lousto,
P. Marronetti, R. A. Matzner, P. Mösta, A. Mroué, D. Müller, B. C. Mundim,
A. Nerozzi, V. Paschalidis, D. Pollney, G. Reifenberger, L. Rezzolla,
S. L. Shapiro, D. Shoemaker, A. Taracchini, N. W. Taylor, S. A. Teukolsky,
M. Thierfelder, H. Witek, Y. Zlochower,
``Error-analysis and comparison to analytical models of numerical waveforms
produced by the NRAR Collaboration'',
Class. Quantum Grav. {\bf 31}, 025012 (2013), arXiv:1307.5307 [gr-qc]
\\

P Ajith, Michael Boyle, Duncan A Brown, Bernd Brügmann, Luisa T Buchman,
Laura Cadonati, Manuela Campanelli, Tony Chu, Zachariah B Etienne,
Stephen Fairhurst, Mark Hannam, James Healy, Ian Hinder, Sascha Husa,
Lawrence E Kidder, Badri Krishnan, Pablo Laguna, Yuk Tung Liu, Lionel London,
Carlos O Lousto, Geoffrey Lovelace, Ilana MacDonald, Pedro Marronetti,
Satya Mohapatra, Philipp Mösta, Doreen Müller, Bruno C Mundim,
Hiroyuki Nakano, Frank Ohme, Vasileios Paschalidis, Larne Pekowsky,
Denis Pollney, Harald P Pfeiffer, Marcelo Ponce, Michael Pürrer,
George Reifenberger, Christian Reisswig, Lucía Santamaría, Mark A Scheel,
Stuart L Shapiro, Deirdre Shoemaker, Carlos F Sopuerta, Ulrich Sperhake,
Béla Szilágyi, Nicholas W Taylor, Wolfgang Tichy,
Petr Tsatsin and Yosef Zlochower,
``Addendum to "The NINJA-2 catalog of hybrid
post-Newtonian/numerical-relativity waveforms for non-precessing black-hole
binaries"'',
Class. Quantum Grav. {\bf 30}, 199401 (2013)
\\

D. Hilditch, S. Bernuzzi, M. Thierfelder, Z. Cao,
\underline{W. Tichy}, B. Bruegmann,
``Compact binary evolutions with the Z4c formulation'',
Phys. Rev. {\bf D88},  084057, 2013,
arXiv:1212.2901 [gr-qc]
\\

\underline{W. Tichy},
``Constructing quasi-equilibrium initial data for binary neutron stars
with arbitrary spins'', Phys. Rev. {\bf D86}, 064024, 2012,
arXiv:1209.5336 [gr-qc]
\\

G. Reifenberger and \underline{W. Tichy},
``Alternatives to standard puncture initial data for binary black hole
evolution'', Phys. Rev. {\bf D86}, 064003, 2012,
arXiv:1205.5502 [gr-qc]
\\

%P. Ajith, Michael Boyle, Duncan A. Brown, Bernd Brügmann, Luisa T. Buchman,
%Laura Cadonati, Manuela Campanelli, Tony Chu, Zachariah B. Etienne, Stephen
%Fairhurst, Mark Hannam, James Healy, Ian Hinder, Sascha Husa, Lawrence E.
%Kidder, Badri Krishnan, Pablo Laguna, Yuk Tung Liu, Lionel London, Carlos O.
%Lousto, Geoffrey Lovelace, Ilana MacDonald, Pedro Marronetti, Satya
%Mohapatra, Philipp Mösta, Doreen Müller, Bruno C. Mundim, Hiroyuki Nakano,
%Frank Ohme, Vasileios Paschalidis, Larne Pekowsky, Denis Pollney, Harald P.
%Pfeiffer, Marcelo Ponce, Michael Pürrer, George Reifenberger, Christian
%Reisswig, Lucía Santamaría, Mark A. Scheel, Stuart L. Shapiro, Deirdre
%Shoemaker, Carlos F. Sopuerta, Ulrich Sperhake, Béla Szilágyi, Nicholas W.
%Taylor, Wolfgang Tichy,
%Petr Tsatsin, Yosef Zlochower
P. Ajith et al., \underline{W. Tichy} [co-author],
``The NINJA-2 catalog of hybrid post-Newtonian/numerical-relativity
waveforms for non-precessing black-hole binaries'',
Class. Quant. Grav. {\bf 29}, 124001 (2012), arXiv:1201.5319 [gr-qc]
\\

P. Marronetti and \underline{W. Tichy},
``Recent Advances in the Numerical Simulations of Binary Black Holes'',
Proceedings of the Department of Energy SciDAC Workshop (2011),
arXiv:1107.3703 [gr-qc]
\\

\underline{W. Tichy},
``Initial data for binary neutron stars with arbitrary spins '',
Phys. Rev. {\bf D84}, 024041, 2011,
arXiv:1107.1440 [gr-qc] 
\\

\underline{W. Tichy} and \'E. \'E. Flanagan,
``Covariant formulation of the post-1-Newtonian approximation to General
Relativity'', Phys. Rev. {\bf D 84}, 044038, 2011,
arXiv:1101.0588v1 [gr-qc]
\\

\underline{W. Tichy} and P. Marronetti,
``A Simple method to set up low eccentricity initial data for moving
puncture simulations'', Phys. Rev. {\bf D 83}, 024012, 2011,
arXiv:1010.2936v2 [gr-qc]
\\

B. J. Kelly, \underline{W. Tichy}, Y. Zlochower, M. Campanelli, B. F. Whiting, 
``Post-Newtonian Initial Data with Waves: Progress in Evolution'',
Class. Quant. Grav. {\bf 27}, 114005, 2010,  arXiv:0912.5311 [gr-qc]
\\

N. K. Johnson-McDaniel, N. Yunes, \underline{W. Tichy}, and B. J. Owen,
``Conformally curved binary black hole initial data including tidal
deformations and outgoing radiation'',
Phys. Rev. {\bf D 80}, 124039, 2009, arXiv:0907.0891 [gr-qc]
\\

\underline{W. Tichy},
``Long term black hole evolution with the BSSN system
by pseudo-spectral methods'',
Phys. Rev. {\bf D 80}, 104034, 2009, arXiv:0911.0973v2 [gr-qc] 
\\

I. Vega, S. Detweiler, P. Diener, and \underline{W. Tichy},
``Self-force with (3+1) codes: a primer for numerical relativists'',
Phys. Rev. {\bf D 80}, 084021, 2009, arXiv:0908.2138 [gr-qc]
\\

\underline{W. Tichy},
``A new numerical method to construct binary neutron star initial data'',
Class. Quant. Grav. {\bf 26}, 175018 (2009), arXiv:0908.0620 [gr-qc]
\\

% B. Aylott, J. G. Baker, W. D. Boggs, M. Boyle,
% P. R. Brady, D. A. Brown, B. Brügmann, L. T. Buchman,
% A. Buonanno, L. Cadonati, J. Camp, M. Campanelli, J. Centrella,
% S. Chatterji, N. Christensen, T. Chu, P. Diener, N. Dorband,
% Z. B. Etienne, J. Faber, S. Fairhurst, B. Farr,
% S. Fischetti, G. Guidi, L. M. Goggin, M. Hannam, 
% F. Herrmann, I. Hinder, S. Husa, V. Kalogera, D. Keppel,
% L. E. Kidder, B. J. Kelly, B. Krishnan, P. Laguna, C. O. Lousto,
% I. Mandel, P. Marronetti, R. Matzner, S. T. McWilliams,
% K. D. Matthews, R. Adam Mercer, S. R. P. Mohapatra, A. H. Mroué,
% H. Nakano, E. Ochsner, Y. Pan, L. Pekowsky,
% H. P. Pfeiffer, D. Pollney, F. Pretorius, V. Raymond,
% C. Reisswig, L. Rezzolla, O. Rinne, C. Robinson,
% C. Röver, L. Santamaría, B. Sathyaprakash, M. A. Scheel,
% E. Schnetter, J. Seiler, S. L. Shapiro, D. Shoemaker,
% U. Sperhake, A. Stroeer, R. Sturani, \underline{W. Tichy}, Y.
% T. Liu, M. van der Sluys, J. R. van Meter, R. Vaulin,
% A. Vecchio, J. Veitch, A. Viceré, J. T. Whelan, Y. Zlochower,
B. Aylott et al., \underline{W. Tichy} [co-author],
``Testing gravitational-wave searches with numerical relativity waveforms:
Results from the first Numerical INJection Analysis (NINJA) project'',
Class. Quant. Grav. {\bf 26}, 165008 (2009), arXiv:0901.4399 [gr-qc]
\\

L. Cadonati et al., \underline{W. Tichy} [co-author],
``Status of NINJA: the Numerical INJection Analysis project'',
Class. Quant. Grav. {\bf 26}, 114008 (2009), arXiv:0905.4227 [gr-qc]
\\

\underline{W. Tichy} and P. Marronetti,
``The final mass and spin of black hole mergers'',
Rapid Communication in Phys. Rev. {\bf D 78}, 081501 (2008),
arXiv:0807.2985 [gr-qc]
\\

P. Marronetti, \underline{W. Tichy}, B. Br\"ugmann, J. Gonzalez, 
and U. Sperhake, ``High-spin binary black hole mergers'',
Phys. Rev. {\bf D 77},064010 (2008), arXiv:0709.2160 [gr-qc]
\\

B. Br\"ugmann, J.A. Gonzalez, M. Hannam, S. Husa, U. Sperhake
and \underline{W. Tichy},
``Calibration of Moving Puncture Simulations'', 
Phys. Rev. {\bf D 77}, 024027 (2008), gr-qc/0610128
\\

B. J. Kelly, \underline{W. Tichy}, M. Campanelli and B. F. Whiting
``Black hole puncture initial data with realistic gravitational wave content'',
Phys. Rev. {\bf D 76}, 024008 (2007), arXiv:0704.0628 [gr-qc]
\\

\underline{W. Tichy} and P. Marronetti,
``Binary black hole mergers: large kicks for generic spin orientations'',
Rapid Communication in Phys. Rev. {\bf D 76}, 061502 (2007),
gr-qc/0703075
\\

P. Marronetti, \underline{W. Tichy}, B. Br\"ugmann, J. Gonzalez, M. Hannam, 
S. Husa, and U. Sperhake,
``Binary black holes on a budget: Simulations using workstations'',
Class. Quant. Grav. {\bf 24}, S43-S58 (2007), gr-qc/0701123
\\

% B. Br\"ugmann, J. Gonz{\'a}lez, M. Hannam, S. Husa, P. Marronetti, 
% U. Sperhake, and \underline{W. Tichy},
% {\em Gravitational Wave Signals from Simulations of Black Hole Dynamics},
% contribution to the 9th Results and Review Workshop of HLRS Computing Center, 
% Stuttgart, Germany,
% Oct. 13--14 2006, published in ``High Performance Computing in Science and 
% Engineering 2006'', Springer, 2006.
B. Br\"ugmann, J. Gonz{\'a}lez, M. Hannam, S. Husa, P. Marronetti, 
U. Sperhake, \underline{W. Tichy},
``Gravitational Wave Signals from Simulations of Black Hole Dynamics'',
contribution to the 9th Results and Review Workshop of HLRS Computing Center, 
Stuttgart, Germany,
Oct. 13--14 2006, published in ``High Performance Computing in Science and 
Engineering 2006'', Springer, 2006.
\\

N. Jansen, B. Br\"ugmann and \underline{W. Tichy},
``Numerical stability of the AA evolution system compared to the ADM and
BSSN systems'', Phys. Rev. {\bf D 74}, 084022 (2006)
\\

\underline{W. Tichy}, 
``Black hole evolution with the BSSN system by pseudo-spectral methods'',
Phys. Rev. {\bf D 74}, 084005 (2006), gr-qc/0609087
\\

N. Yunes, and \underline{W. Tichy}, 
``Improved initial data for black hole binaries by asymptotic matching
of post-Newtonian and perturbed black hole solutions'',
Phys. Rev. {\bf D 74}, 064013 (2006), gr-qc/0601046
\\

N. Yunes, \underline{W. Tichy}, B. J. Owen, and B. Br\"ugmann, 
``Binary black hole initial data from matched asymptotic expansions'',
Phys. Rev. {\bf D 74}, 104011 (2006), gr-qc/0503011.
\\

M. Ansorg, B. Br\"ugmann and \underline{W. Tichy},
``A single-domain spectral method for black hole puncture data'',
Phys. Rev. {\bf D 70}, 064011 (2004), gr-qc/0404056
\\

B. Br\"ugmann, \underline{W. Tichy} and N. Jansen,
``Numerical simulation of orbiting black holes'',
Phys. Rev. Lett. {\bf 92}, 211101 (2004), gr-qc/0312112
\\

\underline{W. Tichy} and B. Br\"ugmann,
``Quasi-equilibrium binary black hole sequences for 
puncture data derived from helical Killing vector conditions'', 
Phys. Rev. {\bf D 69}, 024006 (2004), gr-qc/0307027
\\

\underline{W. Tichy}, B. Br\"ugmann and P. Laguna, 
``Gauge conditions for binary black hole puncture data 
based on an approximate helical Killing vector'', 
Phys. Rev. {\bf D 68}, 064008 (2003), gr-qc/0306020
\\

\underline{W. Tichy}, B. Br\"ugmann, M. Campanelli and P. Diener, 
``Binary black hole initial data for numerical general
relativity based on post-Newtonian data'', 
Phys. Rev. {\bf D 67}, 064008 (2003), gr-qc/0207011
\\

\underline{W. Tichy} and \'E. \'E. Flanagan,
``Angular momentum ambiguities in asymptotically flat perturbed
stationary spacetimes'',
Proceedings of the Ninth Marcel Grossmann Meeting on
General Relativity, edited by V.G. Gurzadyan, R.T. Jantzen and R. Ruffini,
World Scientific, Singapore, p. 1622, (2002)
\\

\underline{W. Tichy} and \'E. \'E. Flanagan,
``Angular momentum ambiguities in asymptotically flat spacetimes which 
are perturbations of stationary spacetimes'', 
Class. Quant. Grav. {\bf 18}, 3995 (2001)
\\

\underline{W. Tichy}, \'E. \'E. Flanagan and E. Poisson,
``Can the post-Newtonian gravitational waveform of an inspiraling binary
be improved by solving the energy balance equation numerically?'',
Phys. Rev. D {\bf 61}, 104015 (2000), gr-qc/9912075 
\\

\underline{W. Tichy} and \'E. \'E. Flanagan, 
``How unique is the expected stress-energy tensor of a massive scalar
field?'',
Phys. Rev. D {\bf 58}, 124007 (1998), gr-qc/9807015 
\\

J. von Delft, D. S. Golubev, \underline{W. Tichy} and A. D. Zaikin,
``Parity-Effected Superconductivity in Ultrasmall Metallic Grains'',
Phys. Rev. Lett. {\bf 77}, 3189-3192 (1996), cond-mat/9604072 
\\



%%%%%%%%%%%%%%%%%%%%%%%%%%%%%%%%%%%%%%%%%%%%%%%%%%%%%%%%%%%%%
\bigskip

{\bf Geförderte Forschungsprojekte:}

\begin{itemize}
\item	NSF Gravitation (PHY-0855315)\\
	Numerical Studies of Compact-Object Binaries\\
	Forschungsmittel \$200000\\
	PI: P. Marronetti (FAU), Co-PI: W. Tichy (FAU)\\
	Laufzeit: 08/01/2009-07/31/2012
\item	NSF Gravitation (PHY-0652874)\\
	Numerical studies of binary black hole dynamics and waveforms\\
	Forschungsmittel \$120000\\
	PI: W. Tichy (FAU), Co-PI: P. Marronetti (FAU)\\
	Laufzeit: 08/01/2007-07/31/2009
\item	NSF Gravitation (PHY-0555644)\\
	Orbiting binary black holes\\
	Forschungsmittel \$60000\\
	PI: W. Tichy (FAU), Co-PI: P. Marronetti (FAU)\\
	Laufzeit: 08/01/2006-07/31/2007
\item	NCSA Teragrid Allocation Account (PHY090095)\\
	Simulations of binary black holes\\
	Rechenzeit: 200000 SU\\
	PI: W. Tichy (FAU)\\
	Laufzeit: 06/10/2009-06/09/2010
\item	NCSA Teragrid Allocation Account (PHY080019)\\
	Binary Black Hole Simulations\\
	Rechenzeit: 30000 SU\\
	PI: W. Tichy (FAU)\\
	Laufzeit: 02/01/2008-01/31/2009
\item	NCSA Teragrid Allocation Account (PHY060052)\\
	Orbiting Black Holes\\
	Rechenzeit: 30000 SU\\
	PI: W. Tichy (FAU)\\
	Laufzeit: 09/08/2006-09/30/2007
\item	NCSA Teragrid Allocation Account (PHY050012)\\
	Constructing astrophysically realistic binary black hole initial data\\
	Rechenzeit: 30000 SU\\
	PI: W. Tichy (FAU)\\
	Laufzeit: 01/2005-01/2006
\item	NCSA Teragrid Allocation Account (PHY050018)\\
	Relativistic Astrophysics: Compact-Object Binaries\\
	Rechenzeit: 105000 SU\\
	PI: P. Marronetti (FAU), Co-PI: W. Tichy (FAU)\\
	Laufzeit: 05/01/2005-04/30/2006
\end{itemize}




%%%%%%%%%%%%%%%%%%%%%%%%%%%%%%%%%%%%%%%%%%%%%%%%%%% 
\bigskip
%\bigskip
%\clearpage

{\bf Vorträge auf Einladung:}

\begin{list}{}{\leftmargin=5em \labelsep=1.5em \rightmargin=-8em \labelwidth=7em}
\item[07/2010]	``Long term black hole evolutions with the BSSN system by
		pseudospectral methods''\\
		{Physikalisch-Astronomische Fakult\"at},
		{Friedrich-Schiller-Universit\"at Jena, Germany}
\item[08/2009]	``Neutron star initial data''\\
		{Physikalisch-Astronomische Fakult\"at},
		{Friedrich-Schiller-Universit\"at Jena, Germany}
\item[10/2008]	``Pseudo-spectral methods''\\
		{FAU SpaceTime seminar, Department of Physics},
		{Florida Atlantic University, Boca Raton, FL}
\item[06/2008]	``Using Post-Newtonian theory to build binary black
                    hole initial data for numerical relativity''\\
		{Post Newton 2008 International Workshop},
		{Friedrich-Schiller-Universit\"at Jena, Germany}
\item[05/2008]	``Black hole evolution by pseudo-spectral methods''\\
		{Department of Physics},
		{University of Florida, Gainesville, FL}
\item[04/2008]	``Binary black holes with spin: predicting the spin of
		    the final black hole''\\
		{FAU SpaceTime seminar, Department of Physics}, 
		{Florida Atlantic University, Boca Raton, FL}
\item[08/2007]	``Black holes and gravitational waves''\\
		{Public Lecture}\\ 
		{Astronomical Society of the Palm Beaches, 
		   West Palm Beach, FL}
\item[05/2007]	``Binary black hole initial data with
		    realistic gravitational wave content''\\
		{Institute for Gravitational
		       Physics and Geometry},
		{Penn State University, State College, PA}
\item[02/2007]	``Binary black hole initial data at the interface between
		    PN theory and numerical relativity''\\
		{NR meets 3PN: A Workshop on the Interface between
                   Post-Newtonian Theory}\\ 
		{and Numerical Relativity},
		{Washington University, St. Louis, MO}
\item[07/2006]	``Approximate binary black hole initial data from
		    matched asymptotic expansions''\\
		{New Frontiers in Numerical Relativity},
		{Albert Einstein Institut, Golm, Germany}
\item[07/2006]	``Constructing binary black hole initial data
		    from approximations''\\
		{Physikalisch-Astronomische Fakult\"at},
		{Friedrich-Schiller-Universit\"at Jena, Germany}
\item[05/2006]	``Binary black hole evolutions with moving punctures''\\
		{Astrophysical Applications of Numerical 
                       Relativity Workshop},
		{Guanajuato, Mexico}
\item[05/2006]	``A gentle introduction to the method of moving punctures''\\
		{FAU SpaceTime seminar, Department of Physics}, 
		{Florida Atlantic University, Boca Raton, FL}
\item[03/2006]	``Approximate binary black hole initial data
		    from matched asymptotic expansions''\\
		{2nd Annual Gulf Coast Gravity Meeting},
		{Florida Atlantic University, Boca Raton, FL}
\item[11/2005]	``Simulations of orbiting black holes''\\
		{Numerical Relativity 2005: Compact Binaries},
		{NASA Goddard Space Flight Center}
\item[07/2005]	``Towards realistic binary black hole initial data''\\
		{Physikalisch-Astronomische Fakult\"at},
		{Friedrich-Schiller-Universit\"at Jena, Germany}
\item[02/2005]	``On the construction of realistic initial data 
		    for binary black hole systems''\\
		{Department of Physics},
		{Florida Atlantic University, Boca Raton, FL}
\item[11/2004]	``Constructing Realistic Initial Data for 
                    Binary Black Hole Systems''\\
		{IGPG Seminar at the Institute for Gravitational
		       Physics and Geometry},\\
		{Penn State University, State College, PA}
\item[05/2004]	``Binary black hole initial data and approximate 
		    helical Killing vectors''\\
		{Center for Gravitational Wave Astronomy},
		{University of Texas at Brownsville, Brownsville, TX}
\item[11/2003]	``Binary black hole initial data sequences
		    derived from helical Killing vector conditions''\\
		{Department of Physics},
		{University of Florida, Gainesville, FL}
\item[06/2003]	``Binary black hole initial data based 
		    on post-Newtonian data'' \\
		{Gravitation: A Decennial Perspective}\\
		{Center for Gravitational Physics and Geometry},
		{Penn State University, State College, PA}
\item[05/2003]	``Post-Newtonian initial data for black hole collisions'' \\
		{Department of Physics \& Astronomy},
		{University of Texas at Brownsville, Brownsville, TX}
\item[12/1999]	``On the uniqueness of the expected stress-energy tensor
		    of a massive scalar field''\\
		{Max-Planck-Institut f\"ur Gravitationsphysik},
		{Albert-Einstein-Institut, Golm, Germany}\\
\end{list}


%%%%%%%%%%%%%%%%%%%%%%%%%%%%%%%%%%%%%%%%%%%%%%%%%%% 
%\clearpage
\bigskip
%%\bigskip

{\bf Vorträge auf internationalen Konferenzen:}

\begin{list}{}{\leftmargin=5em \labelsep=1.5em \rightmargin=-8em \labelwidth=7em}
\item[05/2009] ``A new numerical method for the construction of binary
		neutron star initial data''\\
		{APS April Meeting, Denver, CO}
\item[04/2008] ``Binary black holes with spin: predicting the spin of
                 the final black hole''\\
		{APS April Meeting, St. Louis, MO}
\item[07/2007] ``Kicks due to mergers of spinning black holes''\\
		{GR18/Amaldi 7 conference, Sydney, Australia}
\item[04/2007] ``Efficient binary black hole simulations:
		large kicks for generic spin orientations''\\
		{APS April Meeting, Jacksonville, FL}
\item[04/2006] ``Binary Black Hole Evolutions with Moving Punctures:
		Progress report on final\\
		\ \ orbits and merger''\\
		{APS April Meeting, Dallas, TX}
\item[04/2005] ``Approximate binary black hole initial data from matched
		asymptotic expansions''\\
		{APS April Meeting, Tampa, FL}
\item[06/2004] ``Simulations of orbiting black holes''\\
                {Seventh Eastern Gravity Meeting},
		{Bowdoin College, Brunswick, ME}
\item[09/2003] ``Gauge conditions for binary black hole puncture data
		based on an approximate\\
		\, helical Killing vector''\\
		{Advanced School \ Conference on Gravitational Waves},
		{ICTP, Trieste, Italy}
\item[04/2003] ``Binary black hole initial data based 
		   on post-Newtonian data''\\
		{APS April Meeting, Philadelphia, PA}
\item[01/2002] ``Constructing initial data for black hole inspirals based on
		post-Newtonian data''\\
		{Third EU Network meeting on sources of gravitational
		      waves},\\
		{Southampton University, Southampton, UK}
\item[07/2000] ``General orbits of test particles around a Kerr black
		hole with radiation reaction''\\
		{Ninth Marcel Grossmann Meeting}\\
		{University of Rome {\em La Sapienza}, Rome, Italy}
\item[07/2000] ``General orbits of test particles around a Kerr black
		hole with radiation reaction''\\
		{Third International LISA Symposium},
		{Albert-Einstein-Institut, Golm, Germany}
\item[11/1999] ``Can the post-Newtonian gravitational waveform of an
		inspiraling binary be improved\\
		\ \ by solving the energy balance equation numerically?''\\
		{Ninth Midwest Relativity Meeting},
		{University of Illinois, Urbana-Champaign, IL}
\item[03/1999] ``Coordinate independent formulation of post-1-Newtonian
                 general relativity''\\
		{Third Eastern Gravity Meeting},
		{Cornell University, Ithaca, NY}
\item[03/1998] ``How unique is the expected stress-energy tensor of a
		massive scalar field?''\\
		{Second Eastern Gravity Meeting},
		{University of Syracuse, Syracuse, NY}
\end{list}


%%%%%%%%%%%%%%%%%%%%%%%%%%%%%%%%%%%%%%%%%%%%%%%%%%%%%%%%%%%%%
%\clearpage
%\bigskip
\bigskip

{\bf Lehrtätigkeit:}

\begin{list}{}{\leftmargin=6em \labelsep=1.5em \rightmargin=-8em \labelwidth=9em}
\item[seit 2005]	Vorlesungen an der Florida Atlantic University (FAU):
\\
		Undergraduate Level:
		General Physics I (2009), Intermediate Mechanics (2006)
\\
		Graduate Level:
		Mechanics (2009), Quantum Mechanics I \& II (2005-2010)
\item[ 1996-2000]	Lehrassistent in Physik an der Cornell University:\\
		Honors Physics Sequence: Modern Physics, Quantum Mechanics\\
		Engineering Physics Sequence: Mechanics,
		Optics, Waves and Particles \\
		Non-Major Sequence: Fundamentals of Physics, 
		General Physics 
\item[1995-1996]	Tutor in Physik an der Universität Karlsruhe:\\
	Quantenmechanik, Klassische Mechanik
\end{list}

%%%%%%%%%%%%%%%%%%%%%%%%%%%%%%%%%%%%%%%%%%%%%%%%%%%%%%%%%%%%%%%%
%\clearpage
\bigskip
%\bigskip

{\bf Betreute Studenten:}

\begin{itemize}
\item	Anastasija Cabolova (seit 2008)
\item	George Reifenberger (seit 2008)
\item	Petr Tsatsin (seit 2008)
\item	Joseph Triana (Herbst 2008)
\item	Bereket Ghebremichael (Sommer 2008)
\item	Sean Goldberg (Herbst 2007)
\item	Shawn Westerdale (Sommer 2007)
\item	Nico Yunes (2004-2006)
\item	Matthew Deluca (2005)
\end{itemize}

%%%%%%%%%%%%%%%%%%%%%%%%%%%%%%%%%%%%%%%%%%%%%%%%%%%%%%%%%%%%%
%\bigskip
%\bigskip
%
%{\bf Other Teaching Experience:}\\
%
%\begin{tabular}{ll}
%{1996-2000}	& Teaching Assistant in Cornell Physics Department:\\
%		& Honors Physics Sequence: Modern Physics, Quantum Mechanics\\
%		& Engineering Physics Sequence: Mechanics,
%		   Optics, Waves and Particles \\
%		& Non-Major Sequence: Fundamentals of Physics, 
%		   General Physics \\ 
%		&\\
%%{1998} 	& Grader in Cornell Physics Department:\\
%%   		& Graduate Class in Statistical Physics \\
%%		& \\
%{1995-1996}	& Teaching Assistant at Universit\"at Karlsruhe:\\
%   		& Graduate Classes in Quantum Mechanics and Classical
%		   Mechanics \\
%		& \\
%\end{tabular}



%%%%%%%%%%%%%%%%%%%%%%%%%%%%%%%%%%%%%%%%%%%%%%%%%%%%%%%%%%%%%%%%
\bigskip
%\bigskip

{\bf Andere forschungsverbundene Tätigkeiten:}

\begin{itemize}
\item	Mitentwickler des BAM-Programms zur numerischen 
	Simulation der Einstein Gleichungen
\item	Entwickler des SGRID-Programms zur Simulation der Einstein
	Gleichungen mit pseudospektralen Methoden
\item	Mitglied der Florida Atlantic University SpaceTime (FAUST) Gruppe
\item	Mitglied der American Physical Society (APS)
\item	Kollaboration mit Bernd Br\"ugmann an der
	Friedrich-Schiller-Universit\"at Jena im Bereich der Evolution
	von Neutronensternen und Schwarzer Löcher sowie der 
	Weiterentwicklung des BAM-Programms
\item	Kollaboration mit Steven Detweiler an der
	University of Florida (Gainesville) im Bereich der
	Strahlungsrückwirkung
%\item	Collaboration with Benjamin Owen, Nathan Johnson-McDaniel and
%	Nico Yunes at Penn State University on binary black hole
%	initial data
%\item	Member in the LA Grid (Latin American Grid) Research Community
\end{itemize}


%%%%%%%%%%%%%%%%%%%%%%%%%%%%%%%%%%%%%%%%%%%%%%%%%%%%%%%%%%%%%%%%
\bigskip
%\bigskip

{\bf Erbrachte Dienstleistungen:}

\begin{itemize}
\item	Referee für die folgenden Zeitschriften:\\
	Physical Review Letters, Physical Review D,
	Classical \& Quantum Gravity
\item	Organisator der FAUST-Seminare im Physik-Department (seit 2007)
\item	Vorsitzender der Vorträge über
	``Numerical Analysis of Black Hole Binary Systems''
	beim APS April Meeting 2009 in Denver
\item	Externer Gutachter der National Science Foundation in
	2005, 2006 und 2009
\item	Organisator des Physikkolloquiums an der FAU (Frühjahr 2007 und 2010)
\item	Webmaster des Physik-Departments (seit 2006)
\item	Mitglied in den folgenden Komitees im Physik-Department an der FAU:\\
	Curriculum Committee (seit 2005), 
	Graduate Qualifying Exam Committee (seit 2005),
	Undergraduate Advising Committee (seit 2006)
\item	Vorsitzender des Komitees für General Relativity Faculty Search im
	Physics Department der FAU (2010)
\item	Mitglied im Search Committee for Computer Specialist Position
	in FAU's College of Science (2006)
\item	Judge beim Broward Science Fair 2005
\end{itemize}


%%%%%%%%%%%%%%%%%%%%%%%%%%%%%%%%%%%%%%%%%%%%%%%%%%%%%%%%%%%

% \vspace{0.4in}
% 
% \begin{tabular}{ll}
% {\bf Computer Skills:} & \\ 
%  &  Experience in writing numerical code in C, C++, Fortran \\
%  &  Experience with Cactus and writing Thorns for Cactus\\
%  &  Experience with Linux/Unix system administration \\
% \end{tabular}

%%%%%%%%%%%%%%%%%%%%%%%%%%%%%%%%%%%%%%%%%%%%%%%%%%%%%%%%%%%%%%%%
\bigskip
%\bigskip

{\bf Auszeichnungen und Preise:}

\begin{itemize}
\item	2006 College of Science Researcher of the Year Award in the category
	of Assistant Professor at FAU
\item	1993 Baden-W\"urttemberg Austauschstipendium
%??? \item	Diplom mit Auszeichnung: \\
%	Graduated Summa Cum Laude with highest departmental 
%	honors in physics from Universit\"at Karlsruhe.
\item	1989 Buchpreis der chemischen Industrie
\end{itemize}


%%%%%%%%%%%%%%%%%%%%%%%%%%%%%%%%%%%%%%%%%%%%%%%%%%%%%%%%%%%%%%%%
\bigskip
%\bigskip

{\bf Referenzen:}

\begin{itemize}
\item	Bernd Br\"ugmann \\
	Professor der Physik \\
	Theoretisch-Physikalisches Institut \\
	Abbeanum 211\\
	Friedrich-Schiller-Universit\"at Jena\\
	07743 Jena, Germany \\
	{\bf bernd.bruegmann@uni-jena.de}\\
\item	Pablo Laguna \\
	Professor of Physics and CRA Director\\
	Center for Relativistic Astrophysics\\
	1-63 Boggs Bldg \\
	Georgia Institute of Technology\\
	Atlanta, GA 30332 \\ 
	{\bf plaguna@gatech.edu}\\ 
\item	\'Eanna \'E. Flanagan\\
	Professor of Physics and Astronomy \\
	Center for Radiophysics and Space Research \\
	606 Space Sciences Building \\
	Cornell University \\
	Ithaca, NY 14853 \\
	{\bf flanagan@astro.cornell.edu}\\ 
\item	Steven Detweiler \\
	Professor of Physics \\
	Department of Physics \\
	2071 New Physics Building \\
	University of Florida \\
	Gainesville, FL 32611 \\
	{\bf det@phys.ufl.edu}\\ 
%\item	Abhay Ashtekar \\
%	Eberly Professor of Physics \\
%	Institute for Gravitation and the Cosmos \\
%	316 Whitmore Lab  \\
%	The Pennsylvania State University \\
%	University Park, PA 16802 \\
%	{\bf ashtekar@gravity.psu.edu}\\ 
%%\item	Saul Teukolsky\\
%%	Hans Bethe Professor of Physics and Astronomy \\
%%	Center for Radiophysics and Space Research \\
%%	Space Sciences Building \\
%%	Cornell University \\
%%	Ithaca, NY 14853 \\
%%	{\bf saul@astro.cornell.edu}\\ 
\end{itemize}


\end{document}

